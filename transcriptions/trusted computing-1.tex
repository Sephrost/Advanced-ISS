 - Okay, so welcome back. First of all, we finish the analysis of SSH that we
 stopped at some lectures ago. And we have discussed the SSH in general. Since
 SSH is a tool which is quite frequently used for system management, it is a
 good target for attackers because that is usually permitted even across
 firewalls to permit remote management. So there are several attacks against
 SSH, not the protocol itself, which is quite secure, but against
 implementations. And one of the first that we are looking at is Botan Spy.
 Botan Spy is probably, because there is no confirmation, a tool by CIA as
 based on the information provided by Wikileaks on July 6, 2007 as part of the
 notification of Vault 7 released by Wikileaks. So what is Botan Spy? Botan Spy
 is a malware targeting X-Shell. X-Shell is a Windows SSH client used in USA
 mostly and South Korea. In which way is the attack working? The attack is
 working injecting a malicious DLL into a process, into SSH shell process. The
 purpose is to steal username and password for those connections that are
 authenticated with password or username, private key file name and passphrase
 for public keys authenticated connection. So the purpose is to be able to
 impersonate a valid user. It is designed, this tool, to work within a more
 general tool quite interesting, which is named Shell Term, which creates a
 cover channel with a common control server and offers generic DLL injection
 capabilities. It is a tool that is difficult to detect by malware because it
 does not write anything on the disk apart from the DLL, but the DLL is not
 stored on disk. You remember that the DLL, dynamic link library, is something
 which is executed in memory. So this attack is difficult to detect because
 it's not storing a new DLL on disk, which could be detected, but it's putting
 directly in the memory, in the RAM, random access memory, the new DLL, and
 then it is joining the process with that specific DLL. And for the rest, there
 is direct communication with the common control center. No data written to
 disk, hence, very difficult to detect for any malware. And optionally use it
 also offline. Let's imagine that the system is not connected to the internet
 when that command is executed. In that case, it is more detectable because it
 can write collected data to disk. However, those data are encrypted via AS,
 and later when the system will go back to an internet connection, those
 information will be transferred to the command and control center. Okay, when
 I'm presenting these attacks, it's not just to say, okay, oh, there is this
 possibility, but it is about understanding how we can defend against these
 attacks. What's the problem here, can you tell me? Because remember, every
 time there is a successful attack, we study the attack to understanding how to
 avoid the same mistake. So, can we protect against this CIA attack or not? Or
 is 100\% successful? (silence) Which is the weakness being exploited? Because I
 don't see any zero day vulnerability. I see no programming error. (silence)
 It's not really an external source because here it is not specified, but this
 shell term in some way is executed. So, we assume that there is some
 possibility about that. Maybe, but that may be part of some malware for the
 specific attack, you are right, you're right. If we don't permit execution or
 anything strange, we are safe, but that is a generic thing. For this specific
 attack against X-Shell and SSH, what is the problem which is exploit?
 (silence) No, it's not exploiting that, that is the result. Once this malware
 has hit the machine, if you use password, it will copy the password, but also
 if you use strong authentication, it will copy your private key file and your
 passphrase. So, the purpose is stealing your credentials, either
 password-based or public key-based. But what is the error? There is one huge
 error here. (silence) That is one possibility, but it's not the error. That is
 the way in which the data are exfiltrated. And since we are talking about CIA,
 maybe the CIA is hiding the command and control on Google, and maybe you
 permit access to Google through your firewall. So, that is the way in which it
 is used, the export of the data. But the way in which you collect the data.
 Yeah, so, what can you do to avoid that? (silence) It's a problem of the
 developer. Because when you develop any application, you have the choice. Are
 you using static or shared libraries? If you use a shared library, that is a
 DLL in Windows, you decrease the size of your application, but you run the
 risk that at runtime, your application is linked with the wrong DLL. On the
 contrary, if you link against a static library, your application is bigger,
 but you don't run that risk. So, you, as a developer, can minimize this kind
 of risk by using static libraries, which is not the default nowadays,
 unfortunately. Because people say, no, but we want to minimize the
 applications, less space on disk, when there is an update of the library, it
 will be automatically updated, and so on. (silence) Can you do that? Have you
 done that as a developer? That is not possible, because it's not you. Who is
 linking the DLL with your application? Your application at runtime? Is that
 you? Is that the developer? No. Who is? Give the answer. Who? If you don't
 understand how applications are executed, you have a big problem. The linker,
 no. The linker is the one that, when you have compiled your application, finds
 the appropriate library. If it is static, will insert the code in your
 executable. If it is dynamic, will just say, oh, I need this. And that is,
 when you execute your application, you have the loader, which is a component
 of the operating system, that takes an exit, put it in pages in RAM, looks. If
 there, is there anything missing? Oh, you need this library. Okay, let me see.
 I already have a copy in RAM, so I put a pointer to that library. Oh no, you
 are the first one using the DLL. Let me take the DLL, put in RAM, and put the
 pointer. So it's outside your control. You can just say the name of what you
 need, but you don't have any control about which version of that. No, you can
 specify the version, but you cannot control the actual file, such as computing
 the hash and so on. Okay, so in general, when I show you some attacks, the
 important thing is, yes, we understand that this attack is possible, but then,
 try to find the source of the attack, the origin, the problem, in order to
 avoid that in the future. GearFalcon, probably yet another CIA tool. Again,
 it's listed in the Vault 7 release of Wikileaks. And this time, it's not going
 in Windows, but it's going against the enterprise Linux, that is, Red Hat
 CentOS version. And it is attacking directly the OpenSSH command line client,
 which is the most used in the world. Again, it is based on DLL. It reloads a
 malicious DLL to intercept the plain text traffic before encryption or after
 decryption. So it has just inserted a DLL that is placed after the
 encryption/decryption part. So if you are using username and password, it will
 intercept that. Otherwise, we'll not get your private key, because those are
 not part of the traffic, but we'll get the real traffic, because it is
 controlling the plain text being transmitted on the channel before applying
 the protection protocol of SSH. It uses an encrypted configuration file. So
 it's encrypted, you don't see the content, but you can note that there is an
 extra file on your system. The captured data are stored in yet another
 encrypted file, so another sign that you are infected, and that requires root
 access, but it is not integrated with any command and control center. And
 apparently, it freely reads and writes files on disk. Probably relies on the
 fact that there is rare adoption of multi-malware in Linux. So all these
 things are not noted unless you compute the hash of all the files in your
 system and you compare with the gold ones, with the correct content of your
 disk. But probably, those files are stored in a variable part of the file
 system, in which also that kind of information is not useful. Surprisingly,
 it's not very stealthy, not very much sophisticated, because most people that
 use Linux thinks, "Oh, this is not Windows." So we don't have all the
 vulnerabilities of Windows. Yes, but you still need some kind of protection.
 Same thing that happens with Mac. Against SSH itself and not an
 implementation, the most widely used attack is brute force. When you use
 username and passwords. So we get a false sense of security using a secure
 channel. And some people use, over it, insecure authentication, that is,
 username and reusable password. So the typical brute force attack against such
 an environment is trying all password from a dictionary of well-known
 passwords. Example of an attack. This is an experiment done by some security
 researchers. They put, on September 2016, some servers on purpose, protected
 only with username and password, not public key authentication. One was IPv4
 and one was IPv6, and they were deployed in the cloud. The IPv4 server was
 immediately attacked and conquered in 12 minutes. It was just a server put
 there, no publicity, nothing. So you remember, automatic scanning of all IP
 addresses. You have SSH, let me try, okay? As the password for root was
 password. So they did that on purpose to check. And in a few minutes, that was
 converted to a zombie as part of a botnet. So 12 minutes. After one week, the
 IPv6 server was not even attacked. But that is telling you that IPv6 is not so
 much used and so not much interesting. (upbeat music) (upbeat music) (upbeat
 music) (upbeat music) (upbeat music) (upbeat music) (upbeat music) - Okay, for
 Linux. How can we protect against these brute force attacks? For example, by
 setting a lockout after N authentication failures. Sorry, this account is
 locked. You can use, for example, PAM, PAM2 or PAM fail lock. Or you can do
 another thing. Use a famous program by Daniel Bernstein, TCP wrappers, to
 permit or deny access only to specific hosts or networks. That is something
 similar to a firewall, but it's a local firewall controlling who has the right
 with an IP address. We know that IP addresses can be faked, but if you are
 permitting access only from local machines, it's difficult that someone from
 internet can use those addresses. Or you can use IP tables and implement SSH
 rate control. No more than five connections per minute. So if they are trying
 to try all possible password, they will take forever. That is a possibility.
 Of course, independent of the other things, one important thing that should
 not permit direct access as root. You authenticate as a user, and then
 eventually you run sudo to execute something with administrative privileges.
 Not very strong, but it helps. Change from the default port, don't use 22,
 because that will be the first port which is scanned by an attacker. Use any
 other free number. Or there is another interesting program, which is named
 fail to ban, which can put a limit to the number of failures. After that
 number of failures, that address is banned from connecting to this host for a
 certain amount of time, maybe 15, maybe 30 minutes, maybe one week. You decide
 that is a setting that you may have. Or even much better, if you really want
 to use username and password, then don't use just plain username and password,
 use a second factor authentication, such as Google Authenticator. And of
 course, the best solution is disable password-based authentication. So the
 main applications of SSH are remote interactive access with text-based
 interfaces. Execution of commands, don't forget that, because most people
 think SSH is just, I connect and I have a terminal. No, no, no, you can just
 send a command to be executed remotely. Creation of tunnels, and that we have
 studied, and that we have studied, all with the security properties offered by
 SSH. Some notes, in the past it was an add-on, nowadays it's directly
 available in all platforms. With Windows 11, you already have SSH, for
 example, while in the past was some external add-on, not only the client, but
 also the server. That is, you can remotely connect with SSH to your Windows
 machine. And SSH is not dead, as you can see from the RFCs, not only they are
 quite a lot, but they are continuing to be issued and to be modernized. For
 example, Suite B cryptography suites, SHA-2 integrity, in the past it was just
 SHA-1, and so on. ASGCM, elliptic curve, increase of the minimum recommended
 Diffie-Hellman parameters. So, when you look at the RFCs, it's not only for
 finding information, but also for understanding if the protocol is actively
 maintained, because cryptography makes progress, and we need to adapt the
 protocol to the new algorithm. Okay, and you see that it arrives up to this,
 in which RC4 is deprecated, because now we have much better alternatives. And
 for example, here we have the IDBAR curves. So, it is being aligned with what
 we are doing in TLS 1.3. So, even if it is not of the same complexity as TLS
 1.3, it is still using much of the modern cryptography that we have studied.
 Okay, this finishes this part. Let me now go to a new topic. (upbeat music)
 (upbeat music) (upbeat music) (upbeat music) (upbeat music) (upbeat music)
 (upbeat music) (upbeat music) (upbeat music) (upbeat music) (upbeat music)
 (upbeat music) (upbeat music) Okay, so, this new topic has to do with the
 complexity of the current information systems, which are not composed by
 single elements, but by a wide network with many components and highly
 distributed, okay? So, the topic is integrity and attestation. Attestation
 means understanding and proving if the system is behaving correctly or not for
 generic distributed infrastructures. That can be the cloud, can be
 software-defined networks, can be system hosting network functional
 virtualization, and so on. So, let's consider a typical distributed
 infrastructure to understand the complexity of what we are facing. Okay, we
 have the cloud. Great. Cloud is hosting storage, is hosting computational
 units, and we are all using that. But, in order to use that, you need some
 edge devices to access the cloud. That can be 5G networks, can be routers,
 modems, Wi-Fis, home gateways, and so on. But then, you are accessing those
 resources via personal devices. And in some cases, not even real devices, but
 the Internet of Things, very, very small objects. Desktop, laptops, but also
 smartphone, traffic light remotely controlled, a watch, whatever, okay? So,
 you see that already by components, it's very complex. But then, you have all
 the kind of possible connections. Wired, wireless with different technologies,
 short range, long range, satellite, fiber, whatever. So, it's a real nightmare
 to make secure all these kind of things, okay? So, what are we trying to do?
 We have a problem. Because in the past, we had physical elements. We had the
 hardware. When we were talking about a router, that was a real machine, piece
 of art. When we talk about the firewall, it was the same. Nowadays, there is a
 trend towards softwarization. For example, maybe you have studied in other
 courses, software-defined networking, in which all the basic functions, such
 as filtering, routing, switching, are deployed in software modules. On top of
 a generic hardware, not dedicated. But also, software-defined radio. So, even
 the radio and transmission part can be changed and manipulated via software.
 Network function virtualization. Not only SDN, which is for switching and
 routing, but generic network functions. And then, artificial intelligence.
 Artificial intelligence is a computational load deployed in several places.
 So, most of the things that we use nowadays tend to be software running on
 commodity hardware. A standard platform to minimize the hardware cost. And we
 put the complexity in software, so that we are also able to quickly update it,
 if needed, or change functionality, or add in functionality. Good. So, the
 consequence is, we are becoming more flexible, but also more vulnerable.
 Statistically, the software is more prone to bugs than hardware. Hardware has
 got its bugs, for sure. But not as many as software. Okay. And then, there is
 not only that point of bugs. There is the problem of software updates. You are
 receiving updates. Are those trusted updates? Have they been manipulated? Is
 there a malware in the update that has been transmitted to you? So, the
 problem is, we want it easy, or call it fast, call it flexible. We want it
 cheap, and we want secure. But one of the theorems of security is that you can
 have two properties, not three. And you check what you want. Most people go
 for the first two, and the third one is neglected. That's not good. Okay. So,
 the problem that I'm trying to solve, or to address, at least, is, I got an
 infrastructure, communication, computational, storage, and I want to know if I
 can trust that infrastructure for performing the tasks that I want that
 infrastructure to complete. So, here, the definition is trustworthy. Something
 is trustworthy. It will behave in the expected way. So, you expect something,
 and yes, I am trustworthy. I will surely do that. Which is problematic. Can we
 trust the cloud providers? As hardware, as software, as people managing the
 cloud? Trust in the network providers, and in the edge providers. There is the
 core network, and there is the edge network, and they are connected. In
 general, in addition, given the previous schema, remember that the edge and
 the end user devices typically have very low, and in some cases, not at all,
 any kind of access control who can perform the operations. If you have
 physical access, you can do whatever you want. So, on top of that, we have
 also low cost IoT, where the key is not only IoT. Yes, we want IoT, but we
 want to pay the minimum. That typically implies to save on the security cost,
 because we try to release hardware and software as fast as possible with as
 low effort as possible. So, most IoT devices are completely insecure. And then
 we have also the personal devices, not only in your hands, in which I hope you
 are knowledgeable, but in the hands of your father, your mother, your aunt, of
 ignorant people, in the sense that they ignore the problem and don't have any
 clue how to address that. And so, they become a potential victim, a zombie,
 for example. So, if possible, we do our best to protect all these things. That
 is to avoid the attacks, to block the attacks, but no, it's impossible to
 foresee all possible ways of the attacks. So, we are now adopting a different
 strategy. If we cannot block everything, yeah, we can do something to try
 blocking, but then on top of that, we would like at least to monitor the state
 of the system for early detection. Oh, there is an attack undergoing so that
 we can detect and possibly react to that. Yes, we can use IDS, but that is not
 general. There are several kinds of attacks that an IDS cannot detect. So,
 we're trying to do something better. And something better is summarized by
 these two words. We want to do integrity verification. So, to be able to
 verify and demonstrate that the system has not been altered, that the software
 which is running is the software that we expect, that the configuration of
 that software is the one that we expect. Because, yes, you may be running IP
 tables, but if I was able to change the configuration file for IP tables,
 okay, you have no security. So, remember, not only the software components,
 but also the configuration of those components are important if we want to
 achieve some security. And we are studying that because typically when an
 attacker arrives, the first thing that he's trying to do is injecting new
 software, changing some existing software, changing some configurations, and
 so on. So, this is really one of the hottest thing in this moment in the
 internet. So, let's go for integrity. Integrity of hardware is requiring
 answering to the following question. Am I talking to the right intended node,
 or there is a fake node responding to my questions? Does that node host the
 expected physical components? Because someone may leave your node, but then
 substitute one chip inside that, okay. Maybe you know that in the United
 States, it has been forbidden the usage of Huawei devices. Even of Huawei
 components, hardware components. And one of the fear that they have is that
 one of the components, for example, of a smartphone is the GPS sensor. But
 that GPS sensor can be remotely controlled. So, that could receive information
 by the manufacturer for changing the setting and give you wrong information,
 or even completely disconnect you from the GPS. So, that was the fear of the
 United States when using components produced in a different country, not so
 much trusted for them. And vice versa, of course, for China versus American
 components. Each country is nowadays trying to develop their own components.
 And if you want to see what is the last run of this problem, in the past we
 had Western against Russia and China. Now there is a new installment. You know
 where the new factories for chips are being created? Taiwan, with the China
 blocking. Have you seen, have you read, do you read the newspapers? What's
 happened one week ago? That China isolated completely Taiwan for one week with
 a lot of planes, ships, and so on. So, I don't know for how much Taiwan will
 remain independent. Okay, and we have a big part of the manufacturing of chips
 based on Taiwan, which is really, really risky for all the rest of the world.
 Okay? So, where is the production moving? Also for political reasons, not only
 Taiwan. Okay, we try to protect Taiwan as much as we can. Are you going to die
 for Taiwan? Are you going to the army? Okay, that answers the questions. So,
 what is? And again, read the newspaper. United Arab Emirates. They got a lot
 of money by the petrol, but the petrol will finish at some point in time. They
 need to invest in something different. Oh, the world is requesting chips, and
 there are problems with the current places where chips are manufactured. So,
 some of the largest new manufacturing settings are being deployed in the
 United Arab Emirates. Okay, and that is telling you a long point. Also,
 because the Arabs don't trust the U.S. components, because the U.S. components
 are sold to Israel. Do you want in an Arab missile, a component developed by
 the United States, so it will happen the same that happened with the
 smartphones in Lebanon? You see how technology goes with politics, and that is
 a problem. That is a problem. So, I would like very much to be sure that what
 I have built in my hardware device is really that component, and no one came
 in and substituted it with another component. Last year, in the course before
 this one, it was just named Cybersecurity, we had a very nice talk by an IBM
 engineer that told us that in the cloud manufactured and managed by IBM, they
 have that kind of security. When one element is broken, and so, for example,
 you need to replace a network card in the cloud, the new network card must be
 authenticated. When you insert the card, the system is asking to the card,
 okay, do you contain in your firmware a digital signature by IBM that this is
 a card whose components have been validated by IBM? And additional, for the
 most critical chips, there is a digital signature even inside the chip to be
 checked. That is integrity to the highest degree, because IBM is selling the
 cloud to the governments, so they don't accept normal components. So, this
 integrity is becoming really important. Okay, so, that is integrity for
 hardware. What about software? Yes, for example, we have created TLS, oh yeah,
 we have the proof, we are connected to the right website, good. You know what
 exactly, you know that you have connected to a site that can use that private
 key, but when the data exits from the TLS channel, who is the process using
 that data? Can I have a proof of that, which is going to the right place and
 not to the wrong one? Okay, so, am I talking to the right intended software
 component? Is that component correctly configured? Is the baseline software,
 for example, the TLS, or the drivers of the operating system, the expected
 ones, or they have been changed, manipulated? So, that is, in order to answer
 all those questions, there are two things that need to be considered. And
 today, we start with this first one, which is trusted execution environment.
 Okay, I have a complex system, maybe I cannot be sure about everything, can I
 create a small portion of the system in which my software components are
 executed correctly? So, that is named the T, trusted execution environment.
 So, trusted versus trustworthy, we gave the definition of trustworthy, but now
 trusted is someone or something that you rely upon to not compromise your
 security, but you have no guarantee, you have to trust him. Someone or
 something that will not compromise your security is trustworthy. Okay, so,
 trusted is about how you use something. Trustworthy, sorry, is about whether
 it is safe to use something or not. So, trusted execution environment is what
 you may choose to trust, to rely upon, to execute a sensitive task. And the
 sensitive task is normally named the TA, a trusted application. An application
 that requires to be executed in a protected setting, in a protected
 environment. Hopefully, that application is not only trusted, that you hope is
 not misbehaving. Hopefully, that TA is also trustworthy, will not defeat your
 security. TA is a concept that will, yes. Sorry, let me go. Unfortunately, it
 does not work. If it is trustworthy, then it's trusted. If it is trusted,
 maybe trustworthy. We hope, you see here, hopefully it is trustworthy. So, if
 it is trustworthy, it is trusted, but not vice versa. Because trusted is your
 choice. I need you to make a payment. I trust you to not run away with my
 money. I may be wrong. Trustworthy, if I'm sure that due to some special
 reason, you will not run out with my money. So, trustworthy implies trusted.
 You are trustworthy, then I can trust you. But I can trust you without a
 formal proof that you are trustworthy. OK? We normally use them
 interchangeably, but it's not correct if you go reading or writing a
 scientific paper about that. OK? And normally, we talk about trusted because
 trustworthy means that you have a formal verification, that you have a proof
 that will not create any damage. So, in the general, I also see in this
 definition, trusted execution environment, not trustworthy. It would be nice
 to have that trustworthy. So, the TEs were originally developed for, let's
 say, telephones, smartphones. OK, that was the original setting because that
 is a small setting with limited functionalities that required that kind of
 things, especially when we have moved to the smartphone world because in the
 smartphone world, you have several different apps executing in the same
 environment. And that is very risky because when you are making a payment with
 your smartphone, another app could intercept your payment data, could change
 your payment data, and so on. So, we need, absolutely, on the smartphone, keep
 the apps separated. So, the Open Mobile Terminal Platform was developed
 several years ago, 2006, and that is now part of the GSM Association. The
 first specification, Technical Requirement 0, was security requirements. What
 is our target? 2008, now they developed the TE on top of the technical
 requirements of Level 0. In 2010, it took quite a lot of time, the Global
 Platform was launched to define interfaces and certification. That is, the de
 facto certification body for mobile platforms, for TEs on mobile platforms.
 Then, 2012, ARM, Gemalto, and GE+DE, that's a German company specialized in
 integrated circuits for mobile telephones, they formed Trastonic to create an
 Open TE. ARM, surely you know. You know Gemalto? Gemalto is a French company
 developing smart cards and then evolving that to anything which requires
 hardware-based security, OK? And GE+DE is a German company doing a similar
 job. Then, Global Platform and the Trusted Computer Group, TCG, that we will
 meet a lot of times in this explanation, founded a joint working group
 focusing on TE specification and the use of a TE with another object that we
 will meet, the Trusted Platform Module, TPM. The first major business case
 that originated this thing was the need for Netflix to stream high-resolution
 videos on smartphones and tablets. What does that mean? I'm streaming, you
 have a high-resolution subscription to Netflix. You are streaming that on your
 smartphone. If there is not a trusted environment, you can run another app
 which will copy the movie because it is received there. So you can intercept
 it. And that was a strong business case because Netflix said, no, we will not
 permit streaming to smartphone unless this feature is in place, OK? So quite
 often developments happen because there is an economical, financial
 motivation, and not only because there is a need about security. Then, after
 the Netflix case, financial, enterprise, government, nowadays, automotive,
 Internet of Things are all business cases that are important for TE and TPM.
 This is the general concept of a TE. You have some hardware platform, OK, on
 which you can run two different things. One named Rich Environment, R-E-E,
 Rich Execution Environment, in which you can run whatever you want, typically
 a rich OS. If you are on a smartphone, you can run Android. You can run other
 systems. If you are on a normal computer, you can run Linux, or you can run
 Windows, whatever you want, OK? And you have all the normal applications, no
 limits. Then you have another part, which is the Trusted Execution
 Environment, which, as you can see, goes down to the hardware platform. It
 cannot be just a pure part. It must have some hardware trust ankles. And you
 see that here, there are, for example, hardware keys, secure storage,
 peripherals, such as a screen and keyboard, that can be trusted, T-U-I,
 Trusted User Interface. Because, for example, when I'm entering the data for
 my internet banking, now the whole user interface must be accessible only to
 the internet banking app. All the other apps are disabled. That is a trusted
 user interface, both as display and as input devices, OK? Secure element, so
 hardware secure resources. These must be present in the hardware platform.
 Otherwise, you cannot run a TE. So a TE cannot be run on any CPU. TE can be
 run only on specific hardware platform that satisfy those TR0 requirements,
 OK? In fact, you see that you have some trusted drivers. Trusted, not
 trustworthy. If you go with Professor System, can you demonstrate that that
 does not contain any bug and will only perform the correct function, will
 become trustworthy? In this moment, it's trusted. We hope to have developed
 those drivers correctly, and they will interface with the hardware resources.
 Then you have a trusted core framework, the core for execution. Then you have
 an API internal to the TE that can be accessed, so called, only by trusted
 applications. And now you can run several things. For example, the digital
 rights management that Netflix is asking for, a payment application, some
 corporate applications such as a VPN, and so on. These are the only ones here.
 But then you have a TE communication agent that corresponds to the TE client
 API, OK? There is some kind of communication, and this is one of the most
 risky parts. Because we should check carefully what is permitted through that
 interface. Because that is a risk. You are going from an untrusted world, so
 potentially dangerous, into the trusted one. So typically, these API are very,
 very much limited. For example, you are visiting a website, which is possible.
 At one point, you want to make a payment. So maybe there is an API in which
 you say, OK, please start the payment API connecting to this endpoint. That
 may be acceptable, OK? So this is the general schema. And I hope you have
 understood the concept and the fact there are two environments. So the TE
 typically is not always alone. There is the TE, and there is the RE. One part
 for doing whatever you want, the other for doing something which is very
 limited in scope, but requires very high security. So nowadays, TE is a hot
 topic for one thing, which is named confidential computing. And that is the
 next wave for the cloud. There is a specific consortium that you may go on the
 internet and have a look. It is named the CCC, Confidential Computing
 Consortium, that has been created to create the foundations and the pillar for
 the third kind of protection. Last year in ISS, we talked about the protection
 of data in transit with secure channels, such as TLS, VPN, SSH. Protection of
 data at rest, and then we have encryption of data on the disk. And then we
 said, OK, and then there is protection of data in use. When the data are in
 the RAM accessed by the CPU, can we protect them? Normally, the answer is no.
 You cannot protect that. Confidential computing is about protecting the data
 in use with a guarantee that only a trusted application can access that data.
 OK? So when the data is in use, nobody else can read or write the data. And
 they can be processed only by a duly authorized application. You should
 compare this to the various cryptographic techniques for protecting data at
 rest and data in motion. And that is completely different. Because mostly in
 those cases, you have encryption with some shared keys and so on. Here, you
 need something completely different. You need boundaries. You need walls to
 avoid. This is more about access control rather than encryption. OK? In order
 to be able to provide such a service, we need a root of trust, R-O-T. OK? We
 will say there are many kinds of root of trust, but there is this concept.
 Root of trust is not a specific component. But we name it because it is an
 element whose misbehavior cannot be detected. So that is my root of trust. I
 trust, and if that is not working correctly, everything is screwed up. OK? So
 it should be trusted. That's your choice. Should be also trustworthy. And that
 is up to the developers, to the formal verification, and so on. The root of
 trust is part of another important component, which is named the Trusted
 Computing Base, TCB. This is the set of all hardware, firmware, and eventually
 also software components that are critical to the security. In the sense that
 any vulnerability inside the TCB will break the system security. So you
 understand that the TCB should be minimized. That's why I'm saying hardware,
 OK, we have to use it. Firmware, because hardware without some kind of
 software cannot be used. Then firmware, really the minimum layer to permit
 access to the hardware. Eventually software, but software typically means
 something complex, and I would not like to have that trusted. I would prefer
 to have hardware trusted, maybe firmware trusted, because it's very small. If
 I need to trust also software, well, we are expanding the attack surface,
 because that is the point. OK? So root of trust, TCB, Confidential Computing
 Consortium. So some security principles to follow. The TEE, so the environment
 in which we later can execute the applications, should be part of the device
 secure boot chain based on a root of trust, and should verify code integrity
 during each device boot. We will discuss later in the next lecture, tomorrow,
 what is really a secure boot. But you can guess it is something in which if
 there is any modification, the system will not boot. And that thing has to be
 verified not only when you install the system, but every time you boot the
 system. Additional requirements. We would like hardware-based isolation
 between the rich OS environment and the TEE. Hardware-based, because we don't
 want to trust software. If that separation is implemented in hardware, it's
 much better. Again, because an attacker at that point has to attack the
 hardware and not the software if they want to be able to migrate from
 untrusted to trusted. We have several trusted applications. There are
 different choices. In some TEEs, all the trusted applications execute in one
 environment. They are all together. I would like to have separate containers
 in which each trusted application is executed alone. That is not compulsory.
 It's a good wish. Some TEEs permit that. Other TEEs do not permit that. Secure
 data storage. At some point, the trusted applications will need to permanently
 store some information, and we would like that no one else can access that
 data. Not another trusted application, not a rich application. So secure data
 storage using a hardware uniquely accessible only by the operating system of
 the TEE to prevent unauthorized access and modification and any possibility of
 exploiting the data in another device. You cannot easily migrate that to
 another device by simply swapping, for example, the SD card. And then trusted
 path. It means there is privileged and secure access to peripherals. You can
 imagine fingerprint sensor, display, touchpads, keyboards. They should be
 hardware isolated. Again, I don't want to trust software to provide me that
 protection. From the rich OS environment and controlled only by TEE during
 specific actions. So any application running in the rich part should have no
 visibility or access. Included malware. So maybe the rich part has been
 infected by malware. I don't care. That will affect only the data and the
 information processed in the rich environment. When I start a trusted
 application, they will take control and the malware will not affect them in
 any way. Those are the principles that we should follow in designing a TEE.
 There are several TEEs that have been deployed and developed over the time.
 Let me start with a quick overview. One of the first things that were
 developed is named Intel IPT. Intel Identity Protection Technology. And that
 is based on the fact that maybe you don't know, but when you buy a serious CPU
 by Intel, you are buying actually two CPUs. One for running your program, and
 the other CPU is named Management Engine. And normally, the Management Engine
 is not used. [INAUDIBLE] , but it's not working. [INAUDIBLE] [INAUDIBLE]
 [INAUDIBLE] So, inside the Intel CPU, there is a normal CPU for executing your
 task, and a special CPU named Management Engine. Normally, in user commercial
 devices, the Management Engine is not used. But if you are in a corporate
 environment, where desktops, laptops, smartphones are managed by the company
 and not by the user, the Management Engine is used to execute management tasks
 that can be activated even when your device is switched off, in the sense that
 there is a wake-on-LAN functionality, which they send a special packet via
 Wi-Fi or via Ethernet, and that activates the Management Engine to run
 specific tasks. So, this Management Engine can be used for this feature, for
 running a Java applet on the separate CPU, the Management Engine. So, that
 engine is bound to a physical hardware. And now we have that applet, that Java
 applet, that can do several things. For example, it can generate cryptographic
 keys and store that inside protected memory, not accessible to any program.
 So, for example, there is an interface with a driver created by Microsoft, in
 which the Windows Cryptography API, Windows Copy, can request to generate a
 key, and can request to access a key. Or that Intel IPT was used by one-time
 password generation. There is one product, the Vasco MyDigiPass, that will use
 that for storing the secrets. You remember, one-time password, there is a
 secret and the generator for the passwords. Or for secure pin entering, in
 which no one else can see the pin that you are entering, because the chipset
 is also managing the video. And so it is controlling all those kinds of
 things. So, you see, this is an example of a physically separated TE, because
 the Management Engine is actually a different chip. So, not really a TE that
 can run in parallel with the reach. They are running on two separate CPUs.
 Then, let's go to the other famous TE, ARM. ARM has got some of its CPUs that
 are equipped with the so-called trust zone. The trust zone is a TE. And that
 is made in a way in which the normal CPU bus, which should have 32-bit, is
 extended to a virtual 32-bit, which is signaling whether the application is
 running in secure mode or normal mode, OK? This signal is exposed outside the
 CPU to permit also the creation of secure peripherals and secure RAM. The
 creation, because ARM is not creating the computer. ARM is developing the CPU.
 Then you are the designer of the board. You use ARM. You create the RAM. You
 create the peripherals. And you can use that signal to decide who can access
 that peripheral, that memory, and so on, OK? Potentially, we could have also
 an indicator for which mode the CPU is in. And this is a system which is open,
 well-documented. But the main limitation is that it permits only one secure
 environment. You have a normal environment, and you have a secure environment.
 And all the things that are running in the secure environment are all equal.
 So it means that the separation between the different trusted application in
 ARM trust zone is not implemented by hardware, but it's implemented by the
 software running in the TE, which is a bit less secure than having real
 hardware separation between the different applications. For that reason, there
 is currently efforts of ARM to add a third mode of operation. They have the
 unprotected or rich world. They have the trusted world with trust zone. And
 now they are trying to add another zone for specific features that are the
 attestation features that we will discuss. It's not trivial, because ARM is a
 successful platform. It means that if you change the platform now, you are
 breaking a lot of backward compatibility. So they cannot change the way in
 which trust zone is working. They can add another world for specific
 application, so that only the new CPUs and the newly developed software can
 exploit that new world. So trust zone is not very useful by itself, because it
 permits only to create one secure enclave. It's also named the zone of trusted
 application. So for that reason, Gemalto developed the trusted foundation
 system. G+D developed the Mobicore. And those are the operating system, so the
 operating system running the DE part, that split one secure enclave into
 several ones, essentially through a smart card operating system. So in the DE,
 they insert an operating system that was originally developed for smart cards,
 and can run several applications in parallel, protecting from each other. But
 that is software-based, OK? Not hardware-based, because ARM does not have that
 possibility for hardware protection. So later, they decided to merge forces,
 and Trustonic was developed based on Mobicore, OK? But that is a good
 solution, but requires to pay money. There is a license fee. You have to pay
 G+D. The Trustonic TOS is named KiniB, and it's quite well evolved and
 sophisticated. For example, version 500 has even a 64-bit symmetrical
 multiprocessing for embedded systems. So it's not limited only to low-capacity
 CPUs, but maybe even high CPUs. Then there are other competing efforts. For
 example, if you have a Samsung device, you have Android with the KNOX. KNOX is
 the TE developed by Samsung, which is similar in functionality to Trustonic,
 but has added also the secure boot, so the verification that nothing has been
 changed. Otherwise, the smartphone is blocked, OK? I think-- let me check the
 time that we can-- yes, we can stop here. So you have a 10-minute break. And
 then today, we will not continue with this. But as you may have read on the
 official timetable, today, you will have an exercise session with Andrea Zeni
 about PKI, because next week, you will have the laboratory about PKI, OK? We
 will meet again tomorrow, and we will continue with this topic.
