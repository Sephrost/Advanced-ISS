 to activate the recording. So after considering Trastonic, let's consider some
 other examples of DE. One of the most notable ones, which is still used
 nowadays, which is not oriented to mobile devices, but to laptops and also
 servers, is Intel SGX. Well, SGX stands for Software Guard Extensions. This is
 a feature tightly integrated with the CPU, so it can be considered a hardware
 base. When you buy a CPU by Intel, you can check if that supports SGX or not.
 If it is supported, then it modifies memory management. It means that when you
 declare the existence of an enclave inside the Intel environment with
 protection, so that no other process, not even the ones running at the highest
 level of priority inside the CPU, can access that memory area. So we achieve
 the result that we wished for, to have hardware-protected memory areas between
 the different trusted applications. So since no one can access that area, you
 achieve the result that the enclaves are protected from codes executing in
 other areas, both the general one and the other enclaves. And of course, vice
 versa. Also, that enclave cannot-- And we create an enclave. The Intel SGX
 architecture is performing a measurement. --to that, because measurement is
 one of the key points-- [AUDIO OUT] --for getting-- basically means that we
 compute the hash of everything which is meaningful when we start that thing.
 Typically, it means that you compute the hash of the executable that you are
 loading in that enclave. Of course, this solution can be combined with a
 previous one by Intel, the IPT, for having also trusted display, because SGX
 is something which is limited only to the execution environment. So protection
 of the CPU operations, protection of the memory. It does not create anything
 towards the trusted channels that we want to have towards the device. If in
 this architecture, you want also to have trusted input-output, you need to
 pair it with Intel IPT. SGX version 1 kind of CPUs. It was some years ago. You
 could buy even a laptop or a desktop with Intel SGX-1. After that, they have
 decided-- well, Intel has decided that it was not worthwhile investing on
 user-oriented devices. And now, they have created SGX-2, which is
 unfortunately incompatible with SGX-1, which is only server-oriented. It means
 that SGX-2 is available only on high-end CPUs, such as the Xeon ones that are
 the ones that, in the Intel schema, are for data centers. And notice that, in
 order to create an enclave, you need the special permission. You need to use a
 special library created by Intel. And finally, your code must be signed by
 Intel. Otherwise, in an SGX enclave, which is somehow a bit weird, because you
 need to give your code to Intel. Even if just the executable, you need to give
 it. Otherwise, you won't get the signature. So SGX, if you are really a fan of
 Intel and you like it, it's a good solution. For the rest of the world, there
 are some caveats. And to be careful. Inside this course, when we will go to
 the laboratory, we will use one specific TE, or a general-purpose TE, which is
 named the Keystone. Keystone is not a specific TE. But it is an open source
 framework that permits you to create a TE, customize it. So there are several
 features in a TE. Maybe you don't need all of them. Maybe some of them will
 not execute nicely on your specific computational platform. So in case you
 have an embedded system, you have an IoT with a scarce memory, with a low CPU,
 maybe you don't want to build the whole TE. Maybe your attack model, so the
 risks that you want to protect from, don't call for a specific feature. And in
 that case, you can just customize. You cut out the parts that you don't need.
 So that is important, not only for performance, so that it will perform better
 on your specific environment, but it's also helpful, because if something is
 not present, it cannot be attacked. So it goes towards the target of
 minimizing the trusted computing base. The smallest is the part that you need
 to trust, the better. Because yesterday, we say, you trust. Let's try to
 minimize what needs to be trusted. And in general, Keystone is already
 natively providing one untrusted environment, such as the RE of ARM, plus n
 trusted segregated enclaves. So you don't need to do anything special, like
 running Trustonic, for example, to get separate enclaves. So in the Keystone
 model, you have one part around whatever you like, and then one, two, three,
 four, as many as your computational platform is capable to support. Keystone
 is particularly important nowadays in the case of open source hardware. Maybe
 you have seen that in other courses, but RISC-V is really, really hot
 nowadays, not only in Europe, but across the world, because it is the first
 open source hardware CPU. That is a CPU that you can build and customize by
 your own. The designs are all public. And you can implement that on FPGA, if
 you just want to test that and verify, or in the moment in which your design
 is final and you want to achieve very good performance, you can create that as
 a SOC, system chip, integrated with whatever else you need. The RISC-V is the
 core, because it is providing some hardware features that are interesting.
 Intrinsically, RISC-V is just providing core computational functions. But
 since it is open source, people are continuing IPs. And I don't know if you're
 familiar with this term, but I think it's not in this context internet
 protocol, but means intellectual property. When you put an IP, it means you
 put a module that you have the property, but you want to put it open source.
 And so there are specific IPs for adding other instructions to the core set of
 RISC-V. There are instructions for vector computations. There are instructions
 for artificial intelligence. And there are, of course, crypto extensions. You
 want to have implemented that hardware level in this RISC-V, customized by
 yourself. But that is an addition in order to get better performance if they
 are hardware-based. But there is something that you cannot cancel in RISC-V
 supervisor and user in decreasing order of priority. Machine is the most
 powerful, supervisor a bit less, and the user is the less powerful of the
 three modes. That is also associated with the specific memory management,
 which is named the PMP, Physical Memory Protection, which has its
 hardware-based access control to different pages of memory. That avoids that
 one process can access the memory of another process, either in general or
 specifically during a period of time. And since typically input/output devices
 are memory-mapped, that extends also when you are executing, for example, a
 trusted application that needs to have a trusted connection to the keyboard
 and to the display, you set up a new page protected from everything else and
 give that page exclusively to that trusted application. When the execution
 finishes, you can free that page and everybody can use again those pages and
 those input/output devices. The motivation that brought to the birth of Kiston
 is that the normal TEs are quite rigid and uncustomizable. If you compare with
 the other solutions that we have discussed, Intel SGX is a large software
 stack. Because yes, finally it is executing in one enclave. But in order to
 build your enclave, you need to involve a lot of libraries. That means large
 trusted computing, of course. There is a competing solution by AMD. AMD is not
 implementing SGX. That is a proprietary solution of Intel. They are
 implementing SEV, Secure Encrypted Virtualization, which is again a solution
 to create enclaves as virtual machines with hardware-based encryption and
 support. But again, also in this case, you need to develop that. So again, we
 have a large trusted computing base. And finally, we have already discussed
 the ARM trust zone. And the problem in that case is that you don't have enough
 domains. You have just the untrusted and the trusted one and nothing more. You
 see here the trusted hardware, in which you have the RISC-V cores, the basic
 one for performing normal operations, optional hardware features, if you want
 to add the crypto or artificial intelligence instructions, and so on. And then
 you can add, and you should, a root of trust. We will discuss later how to
 implement a root of trust. So this is the minimum hardware that you need. Then
 in order to minimize the trusted computing base, on top of that, we do not run
 a base operating system or an hypervisor. Keystone is organized as having only
 one application running in machine mode, the highest priority, which is named
 the security monitor. Security monitor is something which is performing access
 control between any call coming from the upper layers towards the hardware. So
 every operation that you want to perform on the hardware is mediated by the
 security monitor, which knows if you are permitted or not. And that's all.
 It's not providing anything else. So if you want any other features, you need
 to develop at the upper layers. And here, you have one untrusted domain in
 which you can do whatever you like, running an operating system. There is a
 Linux, and there is Android available for RISC-V. And that is running, for
 example, in supervisor mode, which is one step above machine, but not yet the
 user mode. User mode is the partitions that run. You can create as many as you
 want enclaves. They are named in Keystone. And each enclave is organized in
 two levels. Since we don't have an operating system here, we need to have
 something similar to an operating system. But not-- we don't need a general
 purpose operating system, for example, with the scheduler, memory protection,
 and so on, because each enclave is running one application. So you just need a
 stripped-down operating system with just the minimal features that you need to
 run that specific application. And that is named the Keystone Runtime. And
 also, in that case, we have several versions of runtimes. Keystone is not
 providing-- well, you have one available for testing. When you go to the
 different versions of operating system, have developed minimal runtimes. And
 that, again, helps. That is not more part of the trusted computing base in
 general, but that is part of the trusted computing base for that specific
 application. And then, on top of that, you develop your trusted application,
 which, in the Keystone language, is named an EAP, Enclave Application. And if
 you want more of that, yes, you can have more of them. This is the general
 Keystone architecture. And I'm not going to the details here, but I'm leaving
 to you these pointers. The PDF documents have already been uploaded to the
 portal, so you can read them. They are a paper published at EURSYS 2020 by the
 authors of Keystone. And then, you have a couple of videos with the original
 presentation and a presentation by the Confidential Computing Consortium,
 because Keystone is one of the things that are continuously being adopted and
 developed by the CCC. So I assume, for the exam, that you have read and
 understood these parts. In case there is something that you read and you don't
 understand, you can come back to us with questions. But I read them by myself,
 and it is quite easy to understand. OK, this is finishing our introduction
 about TE. We have seen, even if just for Keystone, that each TE needs trust to
 one concept that was developed many, many years ago, but it is still actual
 nowadays, which is trusted computing. So before having the confidential
 computing, we had the trusted computing, OK? Now, trusted computing is a part
 of the confidential computing, and there is a specific operation which is very
 important, which is named remote attestation, and they go typically in pair.
 So let me consider what is-- because attackers nowadays try to inject malware
 at the lowest possible level, because the lower you go, the more is-- the
 highest is the chance that you are undetected, OK? If you can inject, then it
 will not be detected, OK? And also, because the lower you go, the biggest is
 the part of the system that you are able to affect and control. So if
 possible, the malware is trying to modify the operating system. Or I can try
 to boot an alternative operating system. Or I can try to modify the boot
 sequence or even the boot loader itself, always with the purpose of starting
 not only your trusted part, but also starting an untrusted part, and that
 remains undetected. [AUDIO OUT] So that means that if we want to trust the
 system, we need to protect the boot system and the operating system, OK? So
 once in the past, all systems-- I don't care if it is a laptop or a server--
 they had the BIOS, the basic input-output system. That was very, very custom
 solutions that were developed by specific hardware vendors, but there was not
 a general purpose one. Those preliminary solutions have led to an improvement
 of BIOS not only as functionality, but also as security. Nowadays, we have
 WIFI, which is a unified environment for firmware. So all the firmwares that
 we use nowadays-- not on smartphones, but on desktop, laptop, servers, and so
 on-- is WIFI. And WIFI has native support for firmware signature and
 verification. So the manufacturer of the platform will sign the firmware, the
 WIFI, and the hardware at boot will verify the signature. And as you can
 imagine, if the signature fails-- that means someone has modified the
 firmware-- boot is stopped. The system will not work. Once we have this
 assurance that the basic level has been executed correctly and that it has not
 been manipulated, as part of that, we have the boot loader, which is now
 trusted because we are sure that no one modified it. Otherwise, boot would
 have failed. And now we can give to the boot loader the task to verify the
 operating system before loading it in order to have trust, at least, in what
 is loaded inside the system. OK, that is going in the direction from the
 rootkits. You know this is a generic term for any kind of tool that permits
 roots access on a machine-- that is, access with the highest privileges. There
 are various forms. We have firmware rootkits that are able to overwrite the
 BIOS or the WIFI, depending on your machine, or the firmware of other
 hardware. Please note that every time you have not a laptop, but typically a
 service-- for example, a board for high-speed internet, that board comes with
 a local processor and with a specific firmware of the CPU. But on the firmware
 of the specific boards, maybe we are attacking the firmware of the ride
 controller. So we don't need only to protect the firmware of the CPU. We need
 to protect any kind of firmware inside our system. The purpose is that, of
 course, the rootkit can start before the operating system. Bootkits-- bootkits
 are those that replace the operating system bootloader so that the node loads
 the bootkit before the operating system. Beware, they are very wise. They will
 not alter your operating system. Otherwise, you will detect. So the operating
 system will perform normal operation. But in memory, in RAM, there is another
 object which is operating. And since it started before the operating system,
 it can create a layer that [INAUDIBLE] made by the operating system below the
 operating system before the hardware. Or kernel rootkits replace a portion of
 the kernel of the operating system so the rootkit can start automatically when
 the operating system loads. And finally, driver rootkits-- that is not to be
 confused with the firmware rootkits. Firmware is firmware. The part which is
 permanently stored in read-only memory, possibly, or flash of that device.
 Driver is part of the operating system. Drivers are loaded at boot. So if I'm
 able to be one of the trusted drivers with the hardware, so any communication
 with that piece of hardware will be manipulated by this driver rootkit. So you
 see that there are several kinds of attacks that are possible and are actually
 being performed. So in most of the cases, people say, OK, but we got
 anti-malware. We have got antivirus. We will surely detect that. But that
 means that you are using software to protect other software. And that simply
 does not work because software that you use itself may be attacked or may
 fail, may have bugs, and so on. So if we really want to achieve a better level
 of security, we need the hardware support to protect software. So we attack in
 the software to attack in the hardware, which is still possible, but the
 attack surface is minimized. You need physical access and so on. That is the
 root of trust. The root of trust is that part of the hardware that you cannot
 verify in any way. But if that part works correctly as expected, then we can
 create the rest of the trust in the system. And we will discuss in a moment
 how we do that. So the root of trust should be part of the TCB because my
 fear, my colleague, whatever you like. So you cannot tell me, oh, the root of
 trust is purely hardware. No, no. You have a piece of hardware. But then, in
 order to use it, you need some piece of software. So the trust of the
 computing base will be composed by the hardware, the software root of trust.
 We don't call it in that way. But anyway, the root of trust, which should be
 minimal. OK, let me give you an example of how we can protect the firmware.
 This is a firmware self-protection, or if you prefer, software root of trust,
 as you will see in a moment. This is an example that I have received by some
 friends that work at HP Enterprise. I don't know if you are aware, the
 famous-- [AUDIO OUT] --split into working towards enterprise with networking
 and servers. And another one, working towards final users-- [AUDIO OUT] --and
 developing that's user-oriented versus corporate-oriented. So they gave me
 this example of how that is implemented. The machines created by HP, so the
 enterprise level, have a signature region at a fixed location in the final
 BIOS image. So you see this was the schema that they developed before UEFI.
 The BIOS image is very small. It's just 16 megabytes. But there is a part that
 is marked as, this is the signature. After the BIOS is being built, because
 the BIOS is customized for each piece of hardware, because it's basic
 input/output. So it depends on which input/output devices you have. And you
 need to customize your BIOS for that. But that happens at the manufacturing.
 You build this, and now you can create your custom BIOS for that. So once the
 BIOS is ready for that specific platform, the SHA-256 hash of the specific
 BIOS regions-- not everything, for example, the signature region must be
 excluded, because it does not yet have a value-- is calculated. These regions
 include static code, code that will never be updated. Sometimes you receive a
 message from your manufacturer, oh, you need to update your firmware. That is
 not the part. Static code is static, will never change, will never be updated
 in the future. The BIOS version information, and the microcode-- I hope you
 know that several CPUs nowadays have some instructions that are really not in
 hardware-based. And other operations that are microcoded. And since microcode,
 to customize the CPUs, are the most sensitive parts. Of course, the part which
 is updated when you receive the message, you need to update your firmware,
 cannot be covered, because that will change over time. But this is already
 enough. OK, once this hash is computed for this platform, this hash is sent to
 a specific server, one very secure machine inside HP Enterprise, which is the
 HP signing server, which returns a signed hash image, which is 32 bytes, plus
 the signature itself, and the certificate. All that things, all those things,
 are copied into the signature region. So when we power on the system, there is
 the first part of BIOS, which is immune, calculates the combined hash of each,
 done always. OK, it's permanent. If that region contains data that are valid
 compared to the hash that has been calculated, then the BIOS compares the
 stored hash and the calculated hash. If both are the same boot continuous,
 otherwise the system is halted. And that is just purely firmware-based. Can we
 do anything better if we have a hardware of trust? Because you can understand
 that if here I'm able, for example, to replace the chip that contains the
 firmware, then I'm done. And that is dangerous. Of course, in most of the
 enterprise-level platforms, the chip is not removable, is soldered on the
 board. But if you are very smart on hardware level, you can replace that. And
 additionally, there is another problem. In many platforms, you have one chip,
 one memory chip, which is actually soldered to the board. And then you have
 one free socket, which is a placeholder for the extension BIOS. Because inside
 this chip, you can perform updates. That is changing part of it, the mutable
 part, not the static ones. But maybe at some point in time, you need to add
 quite a lot of things that don't fit there. So for that reason, some platforms
 have got a free socket in which, if you need, you can place another memory
 chip that will be executed. And this is still part of the BIOS. So the BIOS
 will start. And before giving the control to the operating system, will also
 execute anything which is present in the extension BIOS. And since the
 extension BIOS is mutable, we cannot compute the signature over it. And this
 is another very dangerous attack. So you have also to study what is the
 hardware configuration of your platform in order to decide the kind of attacks
 that can be performed against it. Described here is purely software-based,
 because you see that it is the firmware verifying itself. Full stop. Can we do
 any better? Can we use a hardware root of trust to try to protect the
 firmware? OK. This is still from HP Enterprise. They have developed it over
 the time. So self-verification, described in the previous slide, is based on
 the firmware itself. The static portion verifies the part that can be updated.
 But verification can be implemented by an external chip as well. See the
 picture that I am showing you in a moment and a detailed description in the
 next slide. And this is an example with an old kind of CPU, a Danverton x86
 Danverton CPU, in which you see that from the hardware point of view, I have
 this 60 megabyte memory for the BIOS. This is the one which is [INAUDIBLE]
 here. And you see that it has a multiplexer in front, which attaches that, the
 single SPI bus, to two possible sources. One is the CPU, because it must
 execute the BIOS. But there is an additional chip, which is normally not
 present in your board. This is a crypto microcontroller that has got several
 features, including the ability to talk on the SPI bus and the ability to
 drive this multiplexer. So the decision about which is the source to be taken
 is by the crypto microcontroller. So if power is good, if we have power,
 great. But when we have power, this one is not activated. And when we receive
 the reset, then this microcontroller is driving the BIOS to itself. So this
 microcontroller will verify that the content of the BIOS memory is correct.
 Only if the BIOS memory is correct, then the reset signal will be taken out.
 And so now the CPU will be able to execute the BIOS, because in the same
 moment, I have switched the multiplexer to permit access from the CPU. So now
 you see that the root of trust is this chip. That is the hardware root of
 trust. If that behaves correctly, it's because it has got to do with the
 electric signal, and then it must perform signal to verification and drive all
 these signals in the correct way to keep the CPU stopped until verification
 has passed. But if this chip works correctly, then we can trust that the CPU
 has loaded the correct version of the BIOS. And here is the explanation. So
 the external crypto chip validates the BIOS in the SPI flash post the power
 on. Once validation is successful, only at that moment the x86 CPU will be out
 of reset. Otherwise, it will remain in the reset state. So you have no way to
 have the CPU operational if that verification fails. But there is a problem.
 It is a general purpose one that you buy. But you want to verify the signature
 of Eureka Packard. So that chip is a special chip that has a fusing option. So
 once and only once, you can fuse one public key hash-- not the actual public
 key, the public key hash, which is smaller-- that will be used to verify the
 signature of the hash file stored in the signature region. Because in the
 signature region, you have also the certificate. The certificate contains the
 hash of the public key. So you can compare that hash stored inside the crypto
 controller. For the rest, the validation is self-integrity, except which is
 the external chip, which is doing the hardware route of trust. OK. Once we
 have been able to start the BIOS, we start operations. The thing that we are
 performing is the boot of the operating system, which may happen in several
 different ways. Normal boot, normal security, that is no security. That is
 what you normally do. Secure boot, the firmware verifies the signature and
 will halt the platform if the verification fails. This is mostly
 hardware-based. If you have the crypto chip or software based on the content
 of the BIOS memory, and can verify the system loader, because the operating
 system loader is part of the firmware while the actual operating system is on
 the disk. Or you may have trusted boot. Trusted boot assumes that the first
 part of the boot up to the firmware was OK and performs verification only of
 the operating system components. That is what Windows is doing. When you boot
 Windows, and they need to start the drivers for Windows, when they try to
 start the anti-malware that you have installed on your machine, those things
 need to contain the Microsoft signature. If they are not signed by Microsoft,
 they will not be loaded as part of the operating system. And if those
 verification fails, the operating system will not start. So it's not the BIOS
 or the firmware that will not start. It's the operating system, which is
 different. So you have two levels. When we talk about secure boot, it's the
 part related to the firmware up to the moment in which we load the operating
 system. When you have, on the contrary, trusted boot, we assume that the first
 part was OK, and we are just checking what's happening at the operating system
 level. And of course, this is mostly software-based and verifies the correct
 signature until the point in which the operating system becomes operational.
 But there is a third thing, which will be one of our key points in this
 lecture, which is measured boot. Measured boot is the application of one basic
 concept. When we talk about verify. OK, I trust that you are behaving
 correctly. But please, can I check if you are really behaving correctly or
 not? And measured boot is a procedure that we will discuss in detail, which
 measures-- where measurement, again, is taking the hash-- all the components
 executed from the boot time until to level x. And we will see how to extend
 that as much as possible, because the largest the part covered, the better for
 detecting any kind of problem. The difference with the previous kinds--
 because secure boot and trusted boot can stop operations, either at firmware
 level or operating system level-- the measured boot does not stop anything. It
 is just a detection system-- ops operation-- to an external verifier. So not a
 verifier implemented in the same platform. Because if the platform has been
 manipulated, no one will send an alarm. But measured boot is implemented in
 such a way that even if the platform has been attacked, the result of this
 measurement cannot be faked, cannot be manipulated. So we have something
 inside our system that continuously measures what is happening. And if there
 is-- and you need to set up it-- an external verifier, that verifier may ask,
 how are you? Are you well? Are you fine? What are you doing? And these are
 manipulated in any way. Or, of course, they can be manipulated. But we will
 detect, oh, the answer is not correct. And so that machine has been attacked.
 So the point is that you have an external verifier. And that is important.
 Because given the schema that we have discussed yesterday with that
 infrastructure, if we have a verifier inside those infrastructure, that
 verifier may periodically contact all the elements, look at the reports. If
 the reports are OK, we continue operation. If reports are bad, we decide to
 isolate in some way that node, because it's untrusted. It has become
 untrusted. [AUDIO OUT] OK? This is a general schema to discuss about the boot
 types. We have [AUDIO OUT] the first state immutable. Then we have the second
 stage bootloader, which is the one that [AUDIO OUT] can be [AUDIO OUT] system.
 But please note that before the operating system becomes operational, we have
 two important things that have started before the operating system itself--
 the trusted drivers and the anti-malware. They are before the operating
 system. And then once the operating system is operational, we can run any kind
 of applications. Now, secure boot applies to that part. Secure boot is about
 firmware. [AUDIO OUT] If verification fails, the system is stopped. No boot.
 If, on the contrary, we pass the verification, that part is good. Let's go to
 the next level. Next level is trusted boot, which verifies the trust part of
 the operating system. Again, if that fails, stop. System will not work.
 Otherwise, we can go. But we have measured boot. Measured boot is parallel
 because it is not stopping anything. It's just measuring what is applied, what
 is executed. I have stopped it at this point. But my point is that measured
 boot always starts here. And my purpose is to extend it up, up, up, up, to
 measure everything which is happening inside the system. And the purpose of
 that thing is just measuring. And the measures are stored locally. But then we
 will discuss how those measures are reported in a reliable way that cannot be
 faked to an external verifier. OK? OK, to the specific scheme implemented by
 Windows. If you have Windows 10 or Windows 11, the latest versions, they do
 implement the schema that you see here. We have the secure boot for the Wi-Fi
 part. And in the view of Microsoft, this is stopping any kind of boot kit that
 would go there. That is secure boot. Then Microsoft implements trusted boot.
 But please, secure boot is the task of the manufacturer of your board because
 Microsoft does not develop the firmware. So that part is upon Dell, Lenovo,
 IBM, HP, whatever, Hatcher, who you like. Then comes the part of Microsoft.
 The part of Microsoft is related to trusted boot. They start the operating
 system loader, then the kernel, then the system drivers, then the system
 files, then the Elam drivers. Do you know what is Elam? It's early launch
 anti-malware because a part of the anti-malware will execute as a process in
 user space. But there must be a part which is started before any user process.
 And that is named Elam. All anti-malware software nowadays is composed by
 those two parts. One Elam plus another part. They have also split that in two
 for one reason. The Elam part must be given to Microsoft for signature. So I
 don't want to give to Microsoft all my anti-malware solution. I give you just
 the minimal, the one that in order to correct of trusted boot, which is
 implemented by Microsoft, to stop the attacks via rootkits. Then after these
 things are correctly operating, we can start third-party drivers,
 anti-malware, and finally go to the Windows sign-in so the machine is
 operational. And you see in parallel here the measured boot with an external,
 the verifier, can, if allowed by the access control, ask what is going on,
 what has happened since boot, and so on. So you see that this schema is
 aligned with what I have explained to you in general in the previous slide.
 OK, time for the first break. And then we will continue this. OK, let's
 continue on this topic. So we have seen in general trusted, secure, and
 measured boot. And what you should be aware of is the different
 responsibility. Secure boot is in charge of the manufacturer of the hardware,
 manufacturer of the platform. And so it is involved with the private and
 public key of that manufacturer. Trusted boot of the operating system
 manufacturer. So for example, it's the public and private key of Microsoft.
 Measured boot is any third party. For example, the owner of this device that
 wants to check if it is operating correctly. And the other point is that you
 can have one part without the others. You can have a secure boot without
 trusted boot. Your hardware supports secure boot. Your operating system does
 not support trusted boot. Or you may have trusted boot without secure boot.
 Your hardware does not support secure boot. Pure boot. And so that is why
 trusted boot is never trusted boot. Because it is trusting that the rest is
 working correctly. But we don't know. And measured boot is something that
 requires something inside all those things, which can be enabled or not. And
 we will discuss. Now, most of our discussion will be about measured boot. OK,
 let's continue. And let's go to the origin of all these stories. Several,
 several years ago-- I think it could be 20 years ago-- there was a group that
 proposed trusted computing. And trusted computing is the definition to have
 elements in our computing platforms which are behaving as expected. So please
 note that trusted is not the same as good or secure. The behavior needs to be
 the one. So you have a component. You expect that it behaves in a certain way.
 And you would like to verify that it is actually behaving in that way. So here
 is the concept of attestation. An attestable evidence of the platform state,
 verifiable. So it is an evidence, something which is produced by the platform,
 and that you have a way to verify. And then there is the root of trust. Root
 of trust is something which we cannot verify. It's inherently trusted. Trusted
 computing architecture defines schemes for establishing trust in a platform
 that are based on two things. They are all about identity. Identity of a
 hardware component and identity of software components. Those are the two
 baselines. Specifically, the original implementation of trusted computing
 required a hardware root of trust named TPM, the Trusted Platform Module,
 because that is providing methods for collecting and reporting these
 identities. A TPM used in a computer system reports on the hardware and
 software in a way that allows determination of expected behavior. And from
 that expectation, eventually the establishment of trust. So let's go to the
 trusted computing base. Collection of system resources, hardware and software,
 which is responsible for maintaining the security policy of the system. An
 important attribute of a TCB is the ability to prevent itself from being
 compromised by any hardware or software which is not part of the TCB itself.
 So in that sense, the secure boot and the related firmware that we are seeing
 is a TCB. Because, well, it's not if, which is equivalent. Preventing would
 mean no one can modify me. Great. But as a minimum, I'm not operational if
 someone has modified me. Please note the TPM is not the TCB of a system. TPM
 is a component that allows an independent entity to determine if the TCB has
 been compromised. And in some uses, the TPM can help preventing the system
 from starting if the TCB cannot be properly instantiated. We will see that the
 TPM is measuring everything. And if you have a specific hardware set up, if
 the system will stop. This is not the normal behavior, but may be implemented.
 OK. Root of trust. That is a component. Component that must always behave in
 the expected manner because its misbehavior cannot be detected. So root of
 trust, there are many. They are building blocks for establishing trust in a
 platform. And we have root of trust for measurement, RTM, which is measuring
 and send integrity measurements to the RTS. But we will see it in the
 measurements. And then storing them in some place that, for the moment, is
 named RTS. Usually, the CPU executes the CRTM, core root of trust for
 measurement, at boot as the first piece of BIOS or Wi-Fi code to start the
 chain of trust. Because we don't want anything to go undetected. So the first
 thing that you need to start is the measurement system. And then you can start
 all the rest. Because all the rest will be measured by this component. So it's
 very important that we have a CRTM as first element of the BIOS itself. We
 said that the measures collected by the RTM are stored in RTS. RTS is the root
 of trust for storage, which is a shielded or secured storage. Shielded means
 that no one can modify it. And eventually, it's secured in the site who can
 read that memory. And finally, we have the root of trust for reporting. That
 is another entity, part of the trusted computing base, that securely reports
 the contents of the RTR. So you see, three operations. We have split them,
 rather than having just one thing. The one which is taking the measurements.
 The one which is taking care of storing locally those measures. And then
 another component that, when asked either by a local process-- but that is not
 very meaningful-- better from an external verifier, is able to correctly
 report-- so it will read the data from the RTS-- and report that in such a way
 that it cannot be modified. Our aim is, starting from the lowest level of the
 firmware, create a chain of trust. I trust this. Then I can try that piece,
 that piece, that piece. [AUDIO OUT] --to the application. Software component A
 that measures component B, and stores that measurement in the RTS. Then,
 component B is executed, and measures component C, and stores the measurements
 in the RTS. And so on, and so forth. So we create a chain. We trust A, because
 it is behaving correctly. Great. Please measure, and then start B. And B will
 do the same. So component A, the first of the chain, is typically the CRTN,
 which is part of the TCB. Because from some point, we need to start. And we
 need to trust that. By using RTR, a verifier can securely see from the RTS,
 not to the measure of A. Because A is itself the component, which is
 performing the measurement. So that's why it must be trusted. You cannot
 measure it. B and C can only be trusted if A is trustworthy. That is, if A is
 really behaving correctly. Not only we think it is behaving correctly. So if
 you are able to attack A, all the rest is destroyed. OK, let me give you an
 overview of what is actually the TPM, Trusted Platform Module. First of all,
 we try to minimize cost. So this is an inexpensive hardware component. It must
 cost, as a target, less than \$1. It is available on most servers, laptops,
 desktops, and so on. First property, it must be tamper-resistant, but not
 tamper-proof. Tamper-resistant, it means that it is difficult to compromise
 it, but not impossible. So if you go and read the papers over the internet,
 you will see that the implementation of TPM by various vendors, hardware
 problems related to the implementation. It's not the concept which is bad, but
 it depends, tamper-resistant, to which kind of manipulation do you want to
 resist. And each TPM will create a different kind of resistance, but
 tamper-proof is not a requirement. Even if it is performing cryptographic
 operations, but we will see, it is not in a high-speed cryptographic engine.
 That is another mistake that most people do. Oh, I got a TPM. I can perform my
 RSA signatures with it. Yeah, you can do, but it takes forever. So please,
 don't think about that. It is rather slow. It was not a target to have high
 speed. If you want high speed, buy an HSM. It is certified. You cannot sell a
 component claiming it to be a TPM if it has not passed at least the common
 criteria EAL4+, which is quite a good level. And it's a passive component, so
 it needs to be driven by the CPU. Since it is passive, for example, it cannot
 prevent the boot, but can protect data and securely report them. So the TPM is
 implementing two Roots of Trust. Roots of Trust for storage. Once the data has
 been inserted in TPM, you cannot modify apart from specific operations. And
 Roots of Trust for reporting. The TPM is able to reliably report the content
 of that memory. But it is not the Roots of Trust for measurement that is
 external, because we'll perform the measure and we'll put the results inside
 the TPM. Features. We said secure storage. In which sense RTS is a secure
 storage in the TPM? We will see that that is a special memory, which has only
 one operation, which is not normal in memory, which is named extend. And it is
 an operation which is specifically designed for performing measurement of
 software components. We will discuss that later when we detail the RTS. RTR.
 We said reliable reporting the content. OK. Since the memory is inside, I can
 read. But when I give it outside, how can I do that reliably? With a digital
 signature. The TPM contains a private/public key pair. And every time that you
 are extracting the content of the RTS, that will be extracted and provided to
 you with a digital signature. In that sense, it's reliable. You cannot change
 those data. You cannot create fake data. Because they're private. So you can
 really trust that, yes, these are the correct values that are in the RTS in
 this moment. It contains a hardware-based random number generator. That's
 pretty good. Because you know that wrong, pseudo-random number generators are
 the source of many problems. So having that inside the TPM is definitely a
 plus. It contains various crypto-algorithms. Can perform various hash, MAC,
 symmetric and asymmetric encryption. But it's not a crypto-accelerator. It's
 slow. It can create various cryptographic keys for limited uses. You cannot
 think of putting inside the TPM one million cryptographic keys. Yeah, it can
 create some keys for specific uses that we will discuss. And then it is able
 to perform two operations. One is binding, in which data are encrypted using
 the TPM bind key, specific key, which is a unique LSA key descending from one
 main key, named the storage key. What does that mean? Binding means that if
 you want to decrypt this data, you can do that only on the platform that
 contains that TPM. Because if you move the data to another platform, they are
 encrypted with the key which is not there. So you get physical dependence. The
 encryption can be decrypted only on that piece of hardware, which is very good
 against data theft, but which is also a big problem. Because if that machine
 is broken, your data will never be decrypted anymore in the future. So we've
 got a problem of backup of data that have been bound to a TPM. So generally,
 you don't bound generic data. You bound some specific things that are not
 meaningful if taken to a different platform. Then you have a ceiling. Ceiling
 is another important operation that TPM permits to perform. And ceiling is
 similar to binding. We got another key, the ceiling key, but with a key which
 is inside the TPM. And additionally, ceiling also remembers the state-- and we
 will define in a moment what is actually the state-- of the platform. When you
 want to decrypt-- not decrypt, to unseal data, that means decrypting. But that
 will happen only if the platform is in the same state as it was when data were
 sealed. You can think it very easily of something like that. In the moment in
 which I created the data and I sealed them, I had four applications on. Now I
 want to unseal. If the system is running the same things, the same operating
 system, and so on, great. Data will be released. But if a malware is infected,
 or even if it is not malware, someone else is-- which is not expected, the TPM
 will refuse, say, no, sorry. You are not in the correct state. So it's an
 additional security. It is not enough that you authenticate yourself. Oh, I'm
 the system. Please decrypt. You need also to be running the same software
 configuration that is sealing. Additionally, we said that trusted computing is
 also about identifying the hardware components that we have. In which way we
 identify hardware components? Well, authenticate hardware devices. Because
 each TPM contains a unique and secret endorsement key that was created when it
 was produced. Please sign this challenge. Sign this data with the endorsement
 key. Then you are sure that that node that you are talking to is exactly that
 one. Because that key was generated when the TPM was created. And so it cannot
 be stolen in any way. So you see, there are several interesting features,
 apart from cryptographic acceleration that does not exist. But yet, it is a
 meaningful, secure component, this one. There have been two major versions of
 TPM. The first version that gained the widespread acceptance was TPM version
 1.2. That was a minimal version. With the fixed cache RSA for us. That
 depended upon the manufacturer. It had only one storage hierarchy for the RTS.
 And then one root key, the storage root key, that was an RSA 2048-bit key. And
 hardware identity was available via the built-in endorsement key, which was,
 again, an RSA 2048. And sealing could happen only against the PCR value. These
 PCR values are the registers-- R stands for registers-- where the measurements
 are collected. So basically, you can perform sealing against the measures that
 have been collected. And here, you can see a general generation, RSA key
 generator, SHA-1 hash generator, and encryption, decryption, and signature
 engine, only with RSA. In persistent memory, the endorsement key and the
 storage root key. On the contrary, in the flexible memory, the platform
 configuration registers, PCR, the attestation identity keys-- and we will
 discuss later what they are-- and the storage keys. But TPM has become so
 important that it has continued its development. Currently, if you buy a
 laptop and you want it to run Windows 11, you must have TPM 2.0. There are
 Windows 11 agility. There is SHA-1 for backward compatibility, but we have
 also SHA-256. We have RSA, but we have also a CDSA-256. We can compute HMAC.
 We have IS-128, and the various manufacturers can create other versions with
 more cryptographic. We have distinguished among three key hierarchies, one
 from the platform, one for storage, and one for endorsement. Each of these
 hierarchies may have multiple keys and multiple algorithms. When you want to
 use the TPM, they have started a complex system of authorization or access. In
 the past, there was just one password. If you provide the correct password,
 then the TPM will obey your orders. Now, it has become more sophisticated. It
 can involve other elements. And the success of the TPM is also signaled by the
 various versions. There are platform-specific specifications for PC client.
 That was the first application. Mobile devices, you may have a mobile TPM,
 even if that one is not so successful. But what is really successful nowadays
 is automotive TIN. Many of the newest versions of the onboard electronic
 devices are using a TPM for protection. And many embedded systems are starting
 to be equipped with the TPM for the same reason. Implementations, in the sense
 that TPM 2.0 is a specification implemented. And there are various versions.
 The first, the original one, is I create a chip to be put on your board that
 is a discrete TPM, a dedicated chip that implements functionality in its own
 tamper-resistant semiconductor package. You have a chip protected. But
 hardware manufacturers, such as AMD, Intel, they don't like to have some
 competitor to create chips to be put. You don't need to buy that one. For
 example, STMicroelectronics, Infineon, are good producers of TPMs. No, but why
 buy that device? I will have it inside my CPU. So now, these big manufacturers
 have an integrated TPM. So as part of the system on chip of the CPU, there is
 an TPM as performed, as integrated by Intel. So it's not required, in that
 case, to implement the tamper resistance, because it's not a separate device.
 You must attack the CPU physically if you want to attack the TPM, because it's
 inside that. So when you buy something based on Intel, you should check what
 is the chipset that they are providing, because some-- not all-- some Intel
 chipsets have the TPM integrated inside the chipset itself. Or firmware TPM,
 that is a software-only solution, which in some sense goes against the concept
 that we want hardware root of trust. Some kind of operations, we could put it
 inside the firmware, and we could call it. We don't achieve, of course, the
 same level of security. If you are able to have a trusted execution
 environment, then that firmware TPM should be executed inside the TEE. AMD,
 Intel, and Qualcomm have implemented the firmware TPM, because I said that TPM
 in Intel is not in every chipset. It's only the most expensive ones. If you
 have a low-cost Intel CPU, then they are creating, maybe with SGX or with
 other technology, an enclave where the firmware TPM is executed. TPM. If you
 have virtualization architecture, which is based on a hypervisor, a virtual
 machine dedicated to running the TPM software. So it is like the firmware TPM
 executed in a virtual machine. And then that virtual TPM is made available to
 all the other virtual machines. So it is equivalent to have a firmware TPM,
 but for an hypervisor-based virtual machine manager. And finally, there is a
 purely software TPM, which is the one that you will use in the lab. It is just
 an emulator of the TPM that you run in the user space, just like this, and so
 on. Of course, that is useful only for development purposes. Nothing else.
 These various kind of implementations that fits your trust model. Do you trust
 Intel? OK, then you can have the TPM incorporated. Do you trust the
 manufacturer of the TE inside which the firmware TPM is executed? Then you can
 have a firmware TPM. You don't trust any of them by a chip. So in making a
 choice among all these platforms, if you have the choice, that is the kind of
 thing that you have to decide. As I said, anyway, you have Windows 11. Go and
 check which kind of TPM do you have. That is a thing that you can do by
 yourself. OK, so we said that the TPM 2.0 is not having anymore just one
 hierarchies or keys in storage, but it's got three different hierarchies. A
 platform hierarchy, which is related to the firmware of the platform, and
 contains non-volatile storage for keys and data related to the firmware
 itself, BIOS, WIFI, whatever. Then we have endorsement hierarchy. This is
 devoted to a specific role, which is named the privacy administrator. Again,
 keys and data. But what is the privacy administrator? When it was originally
 developed, the TPM, and the concept of Trusted Computing Group was developed,
 there was a fierce opposition by many cybersecurity researchers. Because they
 say, wait a moment. I ask you the measures. You return the measures signed
 with your RSA private key. So I know that it's coming from you, which is good.
 You cannot lie, which is bad. Because I can know that it is your machine. And
 when you move from one point to another, I can follow you. And I can decide a
 policy which is based on the identity. So you have no more privacy, which is
 very bad. So now, in version 2.0, we have different keys. Keys to identify the
 hardware. Keys to identify the owner. Keys to identify the manufacturer. In
 order to try to protect the privacy of the users. And that is the role of the
 privacy administrator. Decide which kind of keys you want to develop, which
 kind of keys you want to use for the various functionalities. And finally, we
 have the storage hierarchy, which is for the platform. Usually, this is also
 the privacy administrator, but it's not compulsory. In the sense, if you are
 an individual user, yes, you are both the privacy administrator and the user.
 If you are inside the company, typically, the privacy administrator is the IT
 manager of your company. And the user, it's you that are using this machine.
 And in this case, this part only contains non-volatile storage. Well, also the
 first one, for keys and data. Each hierarchy has dedicated authorization. So
 for example, a specific password and an access control policy. And a specific
 seed for generating the primary keys. So we generate completely different
 keys. OK. Now, let's give some more details about those sealing and binding
 that I have just listed as features of TPM. We can have physical isolation for
 securely storing data. So you store those data inside the TPM in the
 non-volatile memory. But that quantity of memory is very limited. So
 typically, you are not storing data. You are storing their primary keys,
 permanent keys. OK. And in order to access that, the TPM implements mandatory
 access control. So according to a specific policy, we decide who can have
 access to those keys. Or since that memory is very much limited and can only--
 We can have cryptographic isolation. We want to protect a large quantity of
 data, a file, the TPM. So the storage is outside the TPM. For example, in the
 magnetic or solid-state disk of the platform. Those things that you are
 storing outside may be data or may be other keys. In any case, we call that
 the blob to be protected, a set of bits, OK? Those bits stay out, but they are
 encrypted with the key, which is controlled by the TPM. So even if those data
 are outside, only with the help of TPM, you will be able to decrypt them. And
 again, also in this case, we need mandatory access control. But it's important
 this difference. Physical isolation, we put those data inside the TPM. And
 cryptographic isolation, we cannot protect the data physically, because they
 don't fit the key which is inside TPM, which more or less is equivalent. OK.
 In the TPM, we have two main categories of objects. Primary keys, those used
 for endorsement or storage, they are derived from one of the primary seeds.
 And the TPM never returns the private part of those keys. You can generate
 them. You can ask the TPM, give me the key certificate for that, but nothing
 more. And those keys can be recreated by using the same parameters, assuming
 that the primary seed has not been changed. That is important, because it will
 permit you to create as many keys as you want, because those keys are not
 permanently stored. I want a key. Use the primary seed for user with a value
 54. That is used in the key generation process. The private part remains
 there. The public key remains out. Then you switch off the machine, and that
 key is lost. Next time that you want to use again that key for which you have
 a certificate, you must just remember the parameters. Use the user seed with
 the value 54. And the same key will be generated. OK. Then, those are for
 keys, primary keys. But then we have keys and sealed data objects at SDO. They
 are protected by a storage parent key, one of the states above. And SPK needed
 in the TPM to load or create a key. Or-- or creating these things comes from
 the random number generator, which is inside the TPM. And again, the TPM
 returns the private part, but returns it protected by the SPK. The private
 part needs to be stored somewhere. So you have the sealed data object, data
 that are outside. You have also the key that is protecting that. But that key
 is protected with the storage parent key. So if you want to decrypt all those
 things, you need to have that TPM. And then there is an area for objects where
 the public area is used to uniquely identify an object, while the private
 area, which is not, of course, exposed, is the object secrets. And that exists
 only inside the TPM. The [INAUDIBLE] area, which is used for storage outside
 the TPM. OK, we have mentioned the fact that the measurements are stored in
 special registers, which are named the PCR. And now we see what are these
 famous PCRs. Platform Configuration Register. That is the way in which the TPM
 is implementing the Root of Trust for storage. And this is the core mechanism
 for recording platform integrity. Why? Because it has some hardware features.
 First of all, it can only be reset at platform reset. So when you switch on
 the machine, or there is a specific pin that if you touch it, then the PCRs
 are reset. But then you need physical access. But that is the only operation
 that you can do. Malicious code cannot take measurements back. Once something
 has been put inside, you cannot say, no, go back. No. You can say, start from
 zero again, touching the pin or rebooting the machine. But otherwise, that
 measurement has been stored. The interesting point is that since there are
 many measurements that we will take, we don't have enough PCRs. We would need
 millions of PCRs for measuring everything which is executed on the server. So
 on the contrary, we accumulate the measurements through the only operation
 which exists. The PCR cannot be used for addition, multiplication,
 subtraction, boot, or touching the pin. And then the PCR can be extended. The
 operation extend has the following syntax. You specify extend with a hash
 function. For example, SHA-256. And it works in this way. You take the old PCR
 value, that is the current value present in the PCR. You concatenate that with
 the digest of new data that you want to add to the PCR. And then you compute
 the hash. The result will be stored in that same PCR. So you see that the PCR
 is containing the concatenation of hashes of everything that has been added.
 This is the extend of PCR number 10 comma the new hash that you want to
 concatenate to all the others. It is a sequence. So the value present in a PCR
 will depend on the order in which you perform the extend operations. If you do
 extend A and then extend B, it's not commutative. It's not the same result as
 extend B and extend A. The important thing is that the PCR value can be used
 to get access to other TPM objects. And if you have Windows 11, now you know
 that it is compulsory to use a BitLocker for the operating system disk.
 BitLocker is the disk encryption of Microsoft. And that means that the
 operating system is not decrypted unless the system is in the same state in
 which we will see that there are several PCRs. And some of them take the
 measures of everything that is happening at that boot. If that PCR contains
 the correct value, then BitLocker will unlock, will decrypt your operating
 system disk, and you can boot Microsoft Windows. Otherwise, it will say, no,
 wrong value. Windows cannot be booted. In that sense, we have a linkage
 between the secure boot and the trusted boot. Because the trusted boot part,
 if we have a TPM, which is now compulsory, will check, has everything gone
 well insofar? Because Microsoft does not trust the manufacturer. You should
 have stopped the platform if something was bad, but verify. And the value of
 the PCR is one way to perform this verification. You have a question? No,
 because that is BitLocker. BitLocker is a disk encryption for a specific
 partition. So you cannot boot that partition where you have Windows, unless
 the system is in a specific state. If you have a partition with another
 operating system, you can boot. It requires a bit of tricks. I will explain
 later. So I had Windows, the entire disk, then I reduced this partition, and
 had the dual boot. And I don't think I had any problems. Yeah. I reduced it.
 So the important thing is that that partition on which you have Windows must
 be encrypted with BitLocker. BitLocker is an encryption system. OK? It means
 that the data that are there can only be decrypted if you have the correct
 key. [INAUDIBLE] I had BitLocker before reducing the partition. For example,
 if I had to reduce the partition using the external tools, so not Windows,
 would I have any problems with BitLocker outside Windows? That is a nice
 question. So I don't have the answer for that. I know that there are some
 issues with BitLocker, because we have [INAUDIBLE] and we decided to make them
 dual boot. And there was a big caution. Beware, you have to save one special
 key for BitLocker before reducing the partition. Because when you will restart
 it, it will detect that something has been modified, and you need to enter
 this special key. So it may depend on which tool you are using and the
 specific platform. Conceptually, it should not, but we should see the
 implementation. But the important point that I'm stressing here is that you
 have one way to protect the operating system. [AUDIO OUT] Which is attached to
 the secure boot correctly. Now we have measurements, because we said that the
 measured boot goes in parallel. So now the secure boot has finished. Should
 have performed everything correctly. But there is also the measurement boot,
 which is telling you what the secure boot has done. And that result is stored
 in a PCR. Now, the part of trusted boot is not even the trusted boot. It's
 simply the fact that the disk is encrypted, keeping into account that PCR
 value. And the disk will not be decrypted. So you don't even start the trusted
 boot if decryption does not happen. Is that clear? So you have an external
 value, which is recording what is happening at secure boot. And since you
 encrypted the disk-- well, actually, you sealed the disk. Because if it would
 be normal encryption, that would be possible. It's sealed. So the key depends
 on the fact that you have the authorization. And the PCR contains a specific
 argument. OK. Now let's concentrate, because we have the components that are
 needed to perform the measured boot. Well, actually, it's time for the second
 break. So let's first have the break. And then we will go inside this part.
 OK, so we discussed the data of secure, trusted boot, the basic features of
 the TPM. And now we see we have a look at how all these things can be
 integrated then to perform attestation and integrity verification. So measured
 boot assumes that you have the TPM. That is important. Maybe a discrete
 component, maybe a firmware component, but you have the functionality of TPM
 available. That means that the first stage bootloader in the boot ROM contains
 the core root of trust for measurement. That is the first thing that must be
 started, because it will measure and store in PCR the value. The value of
 what? The value of the second stage bootloader. So first, we measure to get
 started. That is important, because we don't know if it is trusted or not. So
 the component which is trusted, which is the CRTM, performs the measurement.
 The measurement is put inside the PCR. And then, OK, let's go to the second
 stage. In the second stage, we will repeat this procedure. I will measure the
 operating system and store it. The operating system will perform, potentially,
 the same things for the various applications. That is not standard part.
 You've seen that I put it in green, because for most companies, measure [AUDIO
 OUT] boot stops at the set part that involves application [AUDIO OUT] is
 European projects and bringing inside the IDF as a sensitive work, and which
 is being increasingly adopted. But there is no limit. Once you have this chain
 of trust, you can continue to perform that operation. I measure and load,
 measure and load. And everything is accumulated inside suitable PCRs. Now that
 we have the values inside the PCRs, that we can offer them to an external--
 you can also have an internal verifier. But if that has been compromised by an
 attacker, will not behave correctly. So the process by which an external
 [AUDIO OUT] verifier with remote attestation, because I attest your integrity
 state. And so you have a platform containing the TPM that has performed the
 measured boot. And we have a remote verifier. The first step for the remote
 verifier is to send you a challenge, which is typically a nonce to avoid the
 replay attacks. That's important. Otherwise, someone will copy the response
 and then give it back. Now, the platform will create a data packet containing
 the challenge, so that the response depends on the request. And the value of
 the PCRs-- well, actually, along with the challenge, there is also the list of
 what I am interested. So the platform values of those PCRs that I have
 requested. PCR number 0 has got this value. PCR number 1, blah, blah, blah,
 and so on. And we'll put a nice signature signed with the internal key. The
 device identifier. So that cannot be faked. You know that this is the content
 of the root of trust for storage for this specific device. Now, this thing is
 received. Those are the signed measurements. And you, first of all, you
 validate the signature. You validate it cryptographically. You compute, again,
 the hash. You extract the hash from the signature. You compare. They are the
 same. But you also validate the identity. Is this a valid machine to exist in
 my network? It means that before doing this, the verifier should have recorded
 the list of which machines are permitted to operate in its environment. If it
 is receiving something signed with another public key, no, no, no, no, no. You
 should not be here. So you see that we have a double thing. We are checking
 the software status of that component. And we verify if that component, that
 platform, is expected to be inside my network or is something that someone has
 added. Great. Now, we pass the cryptographic validation. We pass the identity
 validation. We have another point. Are these values the expected ones or not?
 Contain the so-called golden database, which is the values that we expect to
 find in the PCRs for machines in my network. And that is a complex database.
 Because if all the machines are equal from the hardware point of view, great.
 But if you have one Dell, three Lenovo, two HP, some are server, some are PCs,
 well, the values will be variable. That's also why you need to have the
 identity. Ah, OK, you are that laptop Dell 1720, so we expect you to have this
 version of the firmware, this version of the operating system, and so on. So
 the lookup that you need to perform is also identity-based because not all the
 configuration hardware and software, OK? So we check the measurements against
 reference measurements, the so-called golden values. So remote attestation is
 very powerful in the sense that if anything fails, you need another thing,
 which is an alarm. OK, no, sorry. I performed the remote attestation and it's
 not gone well. What should we do now? And now we have the problem of managing.
 The management of remote attestation. We should decide, first of all, which
 things are we attesting. Are we checking the PCR values related only to the
 boot part, so boot attestation, or are we checking periodically when executed,
 which is a dynamic attestation? That depends upon your attack model. If you
 are considering only static attacks, that is, someone has come, has modified
 the boot, has modified the operating system, then the first thing is, OK, I
 will perform a check only once when the machine is switched on. But if you
 consider runtime vulnerabilities, that is, once the machine is operating,
 someone is able, for example, via network to enter the machine and modify the
 things, then checking only at the boot is not a good solution. You need
 periodic attestation. I check now, I check in one hour, I check in two hours,
 and so on. Is that good? Once an hour? So you give a window of attack of one
 hour to the attacker. So you have to decide the periodicity of the operation.
 And that is related to the speed. OK, you can check once every hour. Once it's
 compromising your system, OK, no. But that is also going against some physical
 limit. We see that we have a cycle. I create an ounce, you extract the values,
 you perform a signature, you transmit the results, we check in the database.
 How much time does it take, this thing? And note that the signature operation
 is typically the slowest part of this cycle, because the TPM is not a fast
 cryptographic machine. It is slow. So the signature performed by the TPM is
 typically the slowest part, because then the protocol for the request, maybe
 for the response, depends how many things you are asking to be put in the
 response. And the database lookup typically is quite fast. Currently, the
 periodicity is going-- can be as short as a few seconds. We typically tell our
 customers that we are able to provide attestation every five seconds to stay
 on the safe side, OK? Surely, every five seconds, we can check the status of
 the machine. And for several people, that is enough. But other people say, no.
 In five seconds, if the attack is successful, they are making an airplane
 crash, or this plant exploding, and so on. And that is a limit, not of the
 attestation, but of the physical implementation. We will need a fast [AUDIO
 OUT] if we want to go down that part. Then you have another problem, the
 whitelist generation. Because inside the golden database, you have what are
 the correct values that you expect. If we think to apply this to a cloud
 environment, you remember that in the beginning, we said, OK, we have all
 these nice things. Well, generating a whitelist is quite complex, especially
 if you are considering not only the boot and the operating system, if you are
 considering the applications that are running. Because you must list all the
 possible software components to be executed on any machine. For this reason,
 yes, that is feasible. In the past, we have run a European project named the
 T-Cloud, or Trusted Cloud, in which we demonstrated that it was becoming
 giant. But [AUDIO OUT] we are lucky. Because if, on the contrary, you want to
 apply to limited environments, such as internet of things, edge devices,
 software-defined network, ENFV, those are environments with limited software
 components that are running. And so it is much, much easier to list what is
 the software that you expect to find on each node. This is for you to remember
 that even if we often talk technical in these lectures, then you need security
 management. Because maybe you have a wonderful solution from the technical
 point of view, which is unmanageable. So we are talking about this problem, if
 you want to sell this as a product. Can you manage that? In addition, you may
 want to have in your database labels. For example, we are currently
 maintaining all the possible versions. We label in one way. This is good, that
 is the correct one. This was valid one year ago, so it's old. This is buggy.
 This is a version which is old and was replaced because it contained bugs,
 functional bugs. So in case we find it, it's not optimal, but OK, will not
 function correctly. But then we have [AUDIO OUT] vulnerable. This is an older
 container, the vulnerability. If we find it running on one machine, we should
 immediately stop that machine. So having these labels is, again, very useful,
 but needs, again, management. And then, as I mentioned, we should measure not
 only the software components being executed, but also the configurations. The
 configuration may arrive in several ways. I don't know if in other courses we
 have discussed what is MANO. Yes? No? Management and network orchestration,
 typically in NFV and cloud, you talk about MANO. It's a centralized system
 that decides each node in which way should be configured. Or, more simply, you
 use network management solutions based on netconf or SNMP and so on. So we
 have two problems. One problem is, in the previous slide, I said that each
 component is measuring [AUDIO OUT] the next component, which is [AUDIO OUT]
 beside software components, because I need to measure configurations. And we
 are lucky in the sense that for the operating system up, we can measure also
 files operations. You can measure the content of a file, which is important.
 Which tells you that if the configuration is file-based, then you can measure
 it. For example, you have a file for IP tables. And when IP tables is started,
 that is the configuration. Great. But if you have a hardware device, such as a
 router, in which someone is connecting via netconf and telling the router, you
 should route that [AUDIO OUT] traffic in this way, there is no file there.
 There is in-memory configuration. And that, unfortunately, is dependent to the
 platform. The switches, the routers, the firewalls of one company behave in a
 way which is different from the other. And you need some memory-based
 inspection. That is, for example, one possible topic. We will talk later about
 the thesis. But in case someone of you is interested in these things, and also
 in networking, or anyway, hardware devices, this is something that we are
 investigating, creating some systems that supplement the part that is
 measuring components and files with something that performs in-memory scanning
 and detection in order to get a copy of the current configuration for a
 specific software. For example, you look where IP table, take the hash, and
 you report to that. OK. I mentioned that there are several PCRs. Each
 specification of the TPM-- well, this is TCG. I forgot to mention that. This
 is the Trusted Computing Group, which is the group, the alliance, of companies
 that are behind the Trusted Computing. And if you go there, you will see that
 there is IBM, and Apple, Hewlett-Packard. Most of the companies are there. And
 this is the suggested use for PCR on a client PC. PCR0, platform firmware from
 the system board ROM. But also, the WIFI runtime services and the WIFI boot
 services. That is, PCR0 is extend, extend, extend. And finally, it will have
 the values composed by these things. And you can compare with expected PCR0
 for that platform, that version of the firmware. Let's check the rest. PCR1,
 interfaces from other required specification. ACPI, SMBIOS, and so on. That is
 for various other things. PCR2, drivers loaded. PCR3 is not here. That means
 it's not used for PC. But PCR4 contains the legacy OS loader and the WIFI OS
 loader. It contains the system partition GPT partition table, in case someone
 has played a trick and has changed the partition of your disks. You see, that
 is not software. We are measuring a configuration of the system. PCR7 is the
 secure boot policy. Because a secure boot can be disabled by the user. So we
 want to know if this machine has performed the secure boot or not. Because
 measurement works independently. If there is a secure boot, the machine has
 not booted if it was manipulated. But we want to check that. And finally, PCR8
 plus, so up to them, they are devoted to the operating system. So we can check
 several things. And here is names. You see 8, 15 defined for use by the static
 OS, 16 for debug, and 23 for application support. Typically, there are 24
 PCRs. Now, we said that that was measured boot. And they put the applications
 in green, saying, OK, that can be extended also to applications. But the way
 in which extend to applications is rather tricky. Because we have the
 operating system. It's not tricky performing the measure. It's tricky
 understanding if the result is good or not. Because we don't have one PCR for
 every possible application. So we are continuing to extend, extend, extend. So
 how do we know if the result is good or not? So we have the operating system
 that has been-- since we are proposing is that we use one PCR for all the
 applications that are running. But now we have a problem. If I start first
 application A and then application B, it's different if I start B and then A.
 So the content of that PCR will be a nightmare. I can't predict it. I cannot
 tell you, oh, this is good or this is bad. Because it depends on the number of
 applications that have started, the sequence, and so on. So yes, the operating
 system can perform measure then load. So application number one has got this
 value. Well, actually, first I put the value in the PCR. And then I start.
 Then I start application number three. Then I start application number three
 and so on. So I extend, extend, extend. Now, can we do that? And can we
 measure that? Yes, if you have Linux. If you have Mac or you have Windows,
 forget about it. If you have Linux, yes, we can. And it's named IMA. This is a
 standard component of the Linux kernel. But you just need to enable, when you
 configure your kernel, which is named the Integrity Measurement Architecture.
 And you see clearly that the name is related to what we are studying. It
 extends normal attestation that goes up to the moment in which the operating
 system is started to dynamic execution. So it extends the measurement to the
 applications. And it can perform various operations. Collect is accessed. And
 you see that I mentioned a file. But can also be a configured thing before it
 is accessed. May be accessed for execution. May be accessed for reading. May
 be accessed for writing. Every time there is an access of a file, IMA kicks
 in. Store-- the measurements performed are added to a list, kernel residence,
 named the measurement list. So something that should be protected, because it
 is in kernel memory. But just in case someone is attacking that, we also
 extend with the hash of that measure. Number 10, one of the dedicated to the
 operating system, keeps accumulating the hash of everything which is executed.
 So we have the hash, the global hash in PCR number 10, while in the main
 everything that has been executed. Then, appraise. Appraise is a feature which
 is not needed for measured operations. But I mention it just in case someone
 has got a bright idea on how to use that. That is enforcing local validation
 of a measurement against a good value stored in an extended attribute of the
 file. That means I got an executable. For example, the Python interpreter. And
 they feel that someone has changed it, manipulated, replacing it with a copy
 infected with malware. What I can do is to use an extended attribute of the
 Linux file system to save the hash. And if you have IMA with appraise enabled,
 when Python is executed, before executing, IMA computes the hash, compared
 with the hash stored together with the file. And if they are different, Python
 is not started. You understand that this is weak. Because if I have attacked
 the machine and they have modified IMA, or if I was so good to alter this
 attribute, even if it should not be alterable, but if you add a rootkit or
 something similar, I can perform the operation. And it's a local validation
 which can counter some kind of weak attacks, but it's not very, very strong.
 Maybe it's good. Then, which is protecting those extended attributes, which
 include also the appraisal hash against offline attacks. The collect store are
 the two features that we are discussing now. And we are using that in the
 measurement. So when IMA started, the extension of the UEFI measured boot to
 the OS and application is performed in this way. IMA uses PCR number 10. And
 the initial value that it puts inside is the so-called boot aggregate. That is
 the hash of all PCRs from 0 to 7. They are all related to the UEFI. So we
 start from this value. So we get a link between what we are doing, what was
 done down to avoid that you can do strange attacks. Then, we can configure
 what will be measured using a so-called IMA template. The IMA template is the
 configuration file for IMA. In many cases, we use a standard template named
 IMA-NG. But it can be customized. It is a routine operation. Then, the
 measures are exposed in the kernel security file system. So ASCII runtime
 measurements contain all the things that are being measured. This is an
 example. When you dump that file, you get something like this, PCR number 10,
 template hash, template, the file data hash. You see SHA-1 has got this value.
 Then, the first thing which is executed is /init. Great. It has got this value
 with SHA-256. Then, we add this shared library. Then, we add these other
 shared libraries. Everything which is executed is listed here and extended
 into the PCR number 10. Now, we need to verify that. Because we ask to the
 machine, please give me the PCR number 10. You have it. Oh, what should be the
 correct value? When you have IMA enabled, the measurement lists are not alone.
 But they are accompanied by the measurement list. So that part, which is
 kernel resident, but the PCR 10 value is variable, executed, and their order
 of execution. So we needed to compute if the PCR 10 value returned now is
 correct or not. And we do it this way. We start with myPCR10, which is a
 variable on the verifier, equals 0. Because that is the RSA. Then, we said
 that the first thing is the boot aggregate. So we perform extend with the boot
 aggregate, which is possible because in the response, so we add the values,
 PCR 0, 1, 2, 3, up to 7. So we put them together, and we put it there. Then,
 we go through the list in the same order in which the measures were performed.
 So for each measure M of a component C, first point, since here, we have the
 name of the component. OK, specific machine, alarm. I don't look at anything
 else. Something was executed, which is bad. If the measure of that component
 is different from the gold measure of that component, alarm. Yes, that
 component is permitted, but someone has manipulated it. No good. Otherwise, I
 will extend with these measures my PCR 10, and I will repeat. So I'm
 re-performing the extend, but not in the real EPM, in my memory as the
 verifier. In the end, when I finish scanning the measurement list, if my PCR
 10 is equal to the PCR 10 that was returned in the response, then OK, else
 alarm. You see that in this way, we are able to attest everything from the
 boot, to the operating system, to all the applications, and those
 configurations that correspond to a file stored inside the operating system.
 OK, I think it was a dense lecture, so think about it. And next week, we will
 continue extended data to virtualize the environments and other things like
 that. Have a nice weekend. [AUDIO OUT] you
