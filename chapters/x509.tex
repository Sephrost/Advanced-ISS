\chapter{The X.509 standard and PKI}
\section{Public-key certificate (PKC)}
Lets begin with a definition:
\begin{boxH}
  A \textbf{public-key certificate} is a data structure to securely bind a
  public key to some attributes
\end{boxH}
A PKC is said to "securely bind" a public key to some attributes
because it is signed by a trusted entity, the \textbf{Certificate
Authority}, but other techniques for certification exists (e.g.
blockchain, direct trust and personal signature).\\
The attributes in the certificate are those employed in the
transaction being protected by the PKC, which are difficult to decide
a-priori without knowing the context.\\
PKC are most important to achieve \textbf{non-repudiation} and 
\textbf{digital signature} in a secure way.
\subsection{Key Generation in Public Key Cryptography (PKC)}

Key generation in public key cryptography (PKC) involves complex
algorithms and often relies on random number generators (RNGs) to
ensure strong, unpredictable keys. Once a key is generated, the
private key must be securely protected in various contexts:

\begin{itemize}
  \item \textbf{Storage}: The private key needs to be securely
    stored to prevent unauthorized access.
  \item \textbf{Usage}: The private key must also be protected when
    being used, as it could be exposed during operations.
  \item \textbf{Software Application}: In software environments
    (e.g., web browsers), the trustworthiness of the computing
    platform must be considered, as it may be vulnerable to malware
    or weak implementations. This is true both for the context of
    key generation and use.
  \item \textbf{Dedicated Hardware}: Storing and using keys in
    dedicated hardware, such as a smart card, offers enhanced
    protection but comes with limitations in terms of algorithm
    updates or vulnerability patching( which may be difficult or
    even impossible).
  \item \textbf{Key Injection}: Keys may be generated in software
    and injected into hardware devices, a process that can be useful
    for key recovery, but should be restricted to encryption keys to
    maintain security. This is most important for the private key. 
\end{itemize}

PKC key management requires careful consideration of both security and
operational challenges, especially when dealing with long-term
security mechanisms.

\subsection{Certification architecture}

In Public Key Infrastructure (PKI), several key entities are
responsible for managing the lifecycle of Public Key Certificates
(PKCs):

\begin{itemize}
  \item \textbf{Certification Authority (CA)}: The CA is responsible
    for generating and revoking PKCs. It also publishes PKCs and
    maintains information about their status, such as whether they
    are valid or revoked, or even suspended.
  \item \textbf{Registration Authority (RA)}: The RA plays a
    critical role in verifying the claimed identity and attributes
    of a certificate requestor. It authorizes the issuance or
    revocation of PKCs but does not generate them itself.
  \item \textbf{Validation Authority (VA)}: The VA provides services
    that allow third parties to verify the validity status of a PKC,
    because some responsibilities or the RA may be delegated to it.
    This may involve downloading Certificate Revocation Lists (CRLs)
    or querying an Online Certificate Status Protocol (OCSP)
    responder to check the certificate's status in real-time.
  \item \textbf{Revocation Authority}: Although not an official
    term, this role may be assigned to either the RA or CA. The
    revocation authority is delegated to revoke certificates, often
    on a more urgent basis than their issuance (they are available
    most of the time), ensuring that compromised or invalid
    certificates are promptly rendered unusable.
\end{itemize}

These entities work together to ensure the security and
trustworthiness of PKC-based systems by handling key tasks such as
certificate issuance, verification, and revocation.

\subsection{Certificate generation}
The general
When an actor want to generate a certificate, it must first generate a 
key pair (public and private key). The private key is stored locally
and protected(encrypted) as strong as possible, while the public key 
is sent to the CA with some attributes that help identify the actor.
The CA then requires that the attributes are correct and are
associated with the actor, which requires authentication of the actor.
At this point the CA generates the certificate, signs it with its 
private key and sends it back to the actor while also storing it in 
its repository.\\
As you may noticed, the verification is only on the attributes, and
not of the possession of the private key, which will be discussed
later.

\begin{figure}[h]
  \centering
  \includegraphics[width=0.8\textwidth]{img/x509 certificate
  generation.png}
  \caption{Certificate generation}
\end{figure}


In addition to what just discussed, alternative architectures for
certificate generation exist:

\begin{itemize}
  \item \textbf{Key Pair Generation by RA}: In some architectures, the
    Registration Authority (RA) generates the key pair on behalf of
    the user, obtains the Public Key Certificate (PKC), and securely
    distributes the key pair and certificate to the user via a
    secure device, such as a smart card. This approach is often used
    in large companies where the employees are already known to the
    organization.

  \item \textbf{User Authentication via Code}: Another common method
    involves the user visiting the RA to obtain a code that can be
    used to authenticate their certificate request to the
    Certification Authority (CA). The code is typically calculated
    as a Message Authentication Code (MAC) using the shared secret
    key between the RA and CA:

    \[
      \text{code} = \text{MAC}(K, ID)
    \]

    where \(K\) is a shared secret key between the RA and CA, and
    \(ID\) is the user's identity. The user then submits this code to
    the CA to validate the certificate request.
\end{itemize}

These alternative methods provide flexibility for organizations with
different security needs, allowing for centralized key generation and
secure certificate distribution.

\subsection{X.509 certificates}

X.509 is an ITU-T standard that defines the format of public key
certificates used to verify the identity of a key owner in
cryptographic systems. The development of X.509 certificates has
undergone several versions:

\begin{itemize}
  \item \textbf{v1 (1988)}: The initial version of the standard.
  \item \textbf{v2 (1993)}: A minor update with small improvements.
  \item \textbf{v3 (1996)}: Added extensions and introduced version
    1 of the attribute certificate.
  \item \textbf{v3 (2001)}: Further enhancements, including version
    2 of the attribute certificate, which are not used anymore.
\end{itemize}

X.509 is part of the larger X.500 standard for directory services,
often referred to as "white pages," which are used for managing
information about entities in a structured way(directory services).
X.509 provides a solution to the problem of securely identifying the
owner of a cryptographic key.

The certificates and their structures are defined using \textbf{ASN.1}
(Abstract Syntax Notation One), a standard interface for defining data
structures exchanged in networking environments in a neutral and 
platform-independent way. 

\subsection{PKC Scope}
A Public Key Certificate (PKC) contains information that uniquely
associates a cryptographic key with an entity. This binding between
the key and the entity is ensured by a \textbf{Trusted Third Party
(TTP)}, typically referred to as a Certification Authority (CA), which
digitally signs each certificate to guarantee its authenticity.\\

The liability associated with a PKC may be restricted to specific
applications or purposes, as outlined in the CA's certification
policy. This policy defines the intended usage and limits the scope of
the certificate, ensuring it is applied within the proper legal and
technical contexts.

\subsection{Certificate Policy (CP) and Certification Practice
Statement (CPS)}

According to RFC-3647, "Internet X.509 Public Key Infrastructure
Certificate Policy and Certification Practices Framework," both the
Certificate Policy (CP) and Certification Practice Statement (CPS)
play key roles in defining the use and management of Public Key
Certificates (PKCs):

\begin{itemize}
  \item \textbf{Certificate Policy (CP)}: A CP is a named set of rules
    that defines the applicability of a PKC to a specific community
    or class of applications with common security requirements. It
    establishes the minimum requirements for the issuance and
    management of certificates and can be followed by multiple
    Certification Authorities (CAs). For example, a government CP
    may apply to all certification providers issuing certificates
    for official use.

  \item \textbf{Certification Practice Statement (CPS)}: A CPS outlines
    the specific practices that a CA follows when issuing PKCs.
    While a CP specifies the general rules, the CPS provides
    detailed implementation procedures. Each CA develops its own
    CPS, which is tailored to its operations and describes how it
    meets the requirements set forth in the CP.
\end{itemize}

\subsection{X.500 Directory Service}

The X.500 directory service was the first application to use X.509v1
certificates, providing a framework for managing information about
entities in a network. However, three major problems were encountered
with this early system:

\begin{itemize}
  \item \textbf{Lack of Guarantees on the Quality of the CA}: There were
    no clear policies to ensure the reliability and trustworthiness
    of the Certification Authorities (CAs), leading to concerns
    about the security of issued certificates.

  \item \textbf{Lack of an X.500 Infrastructure}: A fully developed X.500
    infrastructure did not exist, making it difficult to access
    certificates or distribute them effectively across networks.

  \item \textbf{Difficulty in Establishing Certification Paths}:
    Establishing a certification path between two arbitrary users
    was challenging, as the relationships between different CAs were
    not well-defined or standardized.
\end{itemize}

These issues hindered the widespread adoption of X.500 and highlighted
the need for more robust certification frameworks.
\subsubsection{Remedies for x.509v1}
To address those problems, the X.509v1 standard was revised to include 
the following features:
\begin{itemize}
  \item force the semantics in the application or in any case in the
    context external to the certificate (eg: RFC-1422 (PEM))
  \item make the certificate more flexible and expressive(more htan the
    identifier and the key). This was solved in X.509v3.
\end{itemize}

\subsection{RFC-1422}
RFC-1422 introduced a hierarchical certification infrastructure rooted
at the \textbf{Internet Policy Registration Authority (IPRA)}, which
oversaw the certification hierarchy. The framework defined special
types of Certification Authorities (CAs) to manage and enforce
certificate policies:

\begin{itemize}
  \item \textbf{Policy Certification Authority (PCA)}: PCAs were
    responsible for establishing and enforcing the policies used to
    issue certificates. They ensured that subordinate CAs followed
    the defined security practices and policies.

  \item \textbf{Name Subordination}: CAs were required to issue
    certificates within a specific subset of the naming hierarchy,
    ensuring structured and consistent name allocation. For example:
    \begin{itemize}
      \item \textbf{CA n.1}: C=IT (country-level CA for Italy)
      \item \textbf{CA n.2}: C=IT, O=Politecnico di Torino
        (organizational-level CA within Italy)
    \end{itemize}
\end{itemize}

This hierarchical model helped structure the certification
infrastructure, defining clear roles and responsibilities for issuing
and managing certificates.

\begin{figure}[h]
  \centering
  \includegraphics[width=0.8\textwidth]{img/x509 internet per
  hierarchy.png}

  \caption{Internet PEM hierarchy (RFC-1422)}
\end{figure}

Unfortunately this was a failure for many reasons:
\begin{itemize}
  \item the hierarchical infrastructure limits the flexibility
  \item the name subordination introduces undesired limits to the
    assignment of X.500 names
  \item the use of the PCA concept is not flexible in commercial
    applications, where the participation of an operator to take a
    decision is impractical, because no-one was willing to trust a
    global PCA.
\end{itemize}

\section{X.509 Version 3}

X.509 Version 3 is a standard that was completed in June 1996 through
a collaborative effort between ISO/ITU and the IETF. Its primary goal
was to define certificates that are suitable for Internet
applications. This version consolidates all modifications necessary to
extend the definitions of certificates and Certificate Revocation
Lists (CRLs) into a single document.\\
Key features of X.509 Version 3 include:

\begin{itemize}
  \item \textbf{Types of Extensions}:
    \begin{itemize}
      \item \textbf{Public Extensions}: These are defined by the
        standard and made publicly available for anyone to use.
      \item \textbf{Private Extensions}: These are tailored to
        specific user communities and are not publicly available.
        Those are considered as blobs if not understand er and
        discarded
    \end{itemize}

  \item \textbf{Certificate Profile}: This refers to a set of
    extensions designed for a specific purpose, ensuring that
    certificates meet particular requirements for different
    applications. An example of a certificate profile is given by
    RFC-5280, titled "Internet X.509 Public Key Infrastructure
    Certificate and Certificate Revocation List (CRL) Profile,"
    which outlines standard practices for X.509 certificates and
    CRLs.
\end{itemize}

\subsection{Base syntax}

\subsection{Critical Extensions}

In the context of X.509 certificates, extensions can be classified as
critical or non-critical, each affecting the certificate verification
process differently:

\begin{itemize}
  \item \textbf{Critical Extensions}: If a certificate contains an
    unrecognized critical extension, it \textbf{MUST} be rejected
    during the verification process. This ensures that any essential
    information required for proper validation is recognized and
    handled appropriately.

  \item \textbf{Non-Critical Extensions}: Conversely, a non-critical
    extension \textbf{MAY} be ignored if it is unrecognized. This
    allows flexibility in certificate processing, as non-critical
    information does not impede the overall verification of the
    certificate.
\end{itemize}

The responsibility for handling these extensions lies entirely with
the entity performing the verification, referred to as the
\textbf{Relying Party (RP)}. The RP must implement logic to correctly
interpret and respond to both critical and non-critical extensions in
accordance with their definitions.

\subsection{Public Extensions}

X.509 version 3 defines four classes of public extensions that enhance
the functionality and applicability of certificates:

\begin{itemize}
  \item \textbf{Key and Policy Information}: Extensions in this
    class provide additional information regarding the key usage and
    policies applicable to the certificate, guiding how the key
    should be used within specific contexts.

  \item \textbf{Certificate Subject and Certificate Issuer
    Attributes}: These extensions include attributes related to both
    the subject of the certificate (the entity that the certificate
    represents) and the issuer (the Certification Authority),
    enabling more detailed identification and classification.

  \item \textbf{Certificate Path Constraints}: These extensions
    define rules and limitations regarding the certification path,
    ensuring that certificates can only be used in certain contexts
    or under specific conditions, enhancing security within the
    certificate hierarchy.

  \item \textbf{CRL Distribution Points}: This extension indicates
    where the Certificate Revocation List (CRL) can be found,
    providing necessary information for relying parties to check the
    revocation status of the certificate efficiently.
\end{itemize}

\subsection{Key and policy information}

\subsubsection{Authority key identifier}

Key and policy information extensions in X.509 v3 provide crucial
details about the public keys associated with certificates. One
significant component is the \textbf{Authority Key Identifier (AKI)},
which serves the following purposes:

\begin{itemize}
  \item \textbf{Identification of the Signing Key}: The AKI
    identifies a specific public key used to sign a certificate,
    ensuring that the verification process can accurately trace the
    certificate's authenticity back to the correct authority.

  \item \textbf{Identification Methods}: The identification can be
    achieved through:
    \begin{itemize}
      \item A \textbf{key identifier}, typically represented as the
        digest of the public key (PK).
      \item The combination of \textbf{issuer-name} and
        \textbf{serial-number}, allowing a clear reference to the
        issuing CA's key.
    \end{itemize}

  \item \textbf{Usage}: The AKI is particularly useful in scenarios
    where the same CA might utilize multiple keys (e.g., for
    different assurance levels like low and high assurance).

  \item \textbf{Non-Critical Extension}: While the AKI is classified
    as non-critical, its presence can be vital in certain
    applications, especially when building the certificate chain is
    necessary for verifying trust and authenticity.
\end{itemize}

\subsubsection{Subject key identifier}
The \textbf{Subject Key Identifier} (SKI) extension in X.509 v3 serves
to identify a specific public key associated with a certificate. Key
features include:

\begin{itemize}
  \item \textbf{Identification of Public Key}: The SKI uniquely
    identifies a particular public key used in an application. This
    is especially important in scenarios where the public key may be
    updated or replaced over time.

  \item \textbf{Non-Critical Extension}: The SKI is classified as a
    non-critical extension, meaning that while it provides valuable
    identification information, its absence does not necessarily
    prevent the certificate from being considered valid.
\end{itemize}
\subsubsection{key usage}

The \textbf{Key Usage} (KU) extension in X.509 v3 specifies the
application domains in which a public key may be utilized. Key
characteristics include:

\begin{itemize}
  \item \textbf{Application Domain Identification}: The KU extension
    identifies the specific purposes for which the associated public
    key can be employed, ensuring that it is used appropriately
    within defined contexts.

  \item \textbf{Critical or Non-Critical}: The Key Usage extension
    can be classified as either critical or non-critical:
    \begin{itemize}
      \item If marked as \textbf{critical}, the certificate may only
        be used for the specific purposes indicated in the Key Usage
        extension. Any usage outside the defined scopes would render
        the certificate invalid.
      \item If marked as \textbf{non-critical}, the certificate can
        be used more flexibly, potentially allowing usage beyond the
        defined applications without invalidating the certificate.
    \end{itemize}

  \item \textbf{Permitted Cryptographic Operations}: The following
    cryptographic operations can be defined within the Key Usage
    extension:
    \begin{itemize}
      \item \textbf{digitalSignature}: Allowed for both Certificate
        Authorities (CAs) and users.
      \item \textbf{nonRepudiation}: Specifically permitted for
        users.
      \item \textbf{keyEncipherment}: Permitted for users.
      \item \textbf{dataEncipherment}: Allowing encryption of data.
      \item \textbf{keyAgreement}: Involves:
        \begin{itemize}
          \item \textbf{encipherOnly}: Restricting use to
            enciphering operations.
          \item \textbf{decipherOnly}: Restricting use to
            deciphering operations.
        \end{itemize}
      \item \textbf{keyCertSign}: Specifically permitted for
        Certificate Authorities (CAs).
      \item \textbf{cRLSign}: Also permitted for Certificate
        Authorities (CAs) for signing Certificate Revocation Lists
        (CRLs).
    \end{itemize}
\end{itemize}

\subsubsection{Private key usage period}

The \textbf{Private Key Usage Period} extension in X.509 v3 specifies the
time frame during which the associated private key may be used. Key
features include:

\begin{itemize}
  \item \textbf{Usage Period Definition}: This extension defines the
    period during which the private key can be actively utilized,
    helping to enforce time-based restrictions on key usage.

  \item \textbf{Non-Critical Extension}: The Private Key Usage
    Period extension is always classified as non-critical, meaning
    that its absence does not invalidate the certificate. However,
    the information it provides can be important for managing key
    lifecycles.

  \item \textbf{Usage Discouraged}: While this extension is
    available, its use is generally discouraged in practice.
    Organizations often prefer to manage key lifetimes through other
    means, such as regular key rotation, rather than relying on a
    specified usage period within the certificate.
\end{itemize}

\subsubsection{Certificate policies}

The \textbf{Certificate Policies} extension in X.509 v3 outlines the
specific policies that were adhered to during the issuance of the
certificate and defines the purposes for which the certificate can be
utilized. Key aspects include:

\begin{itemize}
  \item \textbf{Policy Listing}: This extension provides a
    comprehensive list of the policies that govern the certificate
    issuance process, ensuring clarity regarding its intended use.

  \item \textbf{Indication Methods}: Certificate policies can be
    indicated using various formats, including:
    \begin{itemize}
      \item \textbf{Object Identifier (OID)}: A unique identifier for
        the policy.
      \item \textbf{Uniform Resource Identifier (URI)}: A link to a
        location where the policy can be reviewed.
      \item \textbf{Text Message}: A textual description of the
        policy.
    \end{itemize}

  \item \textbf{Critical or Non-Critical}: The Certificate Policies
    extension can be classified as either critical or non-critical:
    \begin{itemize}
      \item If marked as \textbf{critical}, the certificate must only
        be used in accordance with the specified policies; otherwise,
        it may be considered invalid.
      \item If marked as \textbf{non-critical}, the certificate can be
        used more flexibly, although the policies still provide
        guidance for its intended use.
    \end{itemize}

  \item \textbf{Support for Authentication and Authorization}: The use
    of this extension can enhance not only the authentication of users
    and entities but also facilitate authorization processes,
    providing a clearer understanding of the permissible actions
    associated with the certificate.
\end{itemize}

\subsubsection{Policy mappings}

The \textbf{Policy Mappings} extension in X.509 v3 establishes a
correspondence between policies across different certification
domains. Key aspects include:

\begin{itemize}
  \item \textbf{Mapping of Policies}: This extension indicates how
    policies from one certification authority (CA) correspond to
    policies from another, facilitating interoperability and trust
    among different certificate frameworks.

  \item \textbf{Presence in CA Certificates}: The Policy Mappings
    extension is typically present only in CA certificates, allowing
    certification authorities to define and communicate
    relationships between their policies and those of other
    authorities.

  \item \textbf{Non-Critical Extension}: The Policy Mappings
    extension is classified as non-critical, meaning its absence
    does not invalidate the certificate. However, it serves as an
    important tool for enhancing the understanding and usability of
    certificates across different domains.
\end{itemize}

\section{Private extensions}

The \textbf{Private Extensions} in X.509 v3 allow for the definition
of extensions that are specific to a particular user community,
facilitating tailored applications within closed groups. Key aspects
include:

\begin{itemize}
  \item \textbf{Definition of Private Extensions}: Private
    extensions are custom extensions that can be defined for use by
    a specific community or group of users, enabling flexibility and
    specificity in certificate applications.

  \item \textbf{Examples from IETF-PKIX}: The Internet Engineering
    Task Force Public Key Infrastructure (IETF-PKIX) has defined
    three notable private extensions for the Internet user
    community:
    \begin{itemize}
      \item \textbf{Subject Information Access}: Provides
        information about how to access subject-related data.
      \item \textbf{Authority Information Access}: Offers details on
        how to access information related to the certificate
        authority.
      \item \textbf{CA Information Access}: Supplies information
        about accessing resources associated with the CA.
    \end{itemize}
\end{itemize}

